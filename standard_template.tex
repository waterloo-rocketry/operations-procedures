\documentclass[letter]{article}

%% margins and fonts
\usepackage[margin=1in]{geometry}
\renewcommand{\familydefault}{\sfdefault}

% begin package imports -- {{{
\usepackage{amsmath}
\usepackage{graphicx}
\usepackage[dvipsnames]{xcolor}
\usepackage{tabto}
\usepackage{titling}
\usepackage[iso,german]{isodate}
% end package imports -- }}}

% begin set section title styles -- {{{
\usepackage{titlesec}
\titleformat{\section} {\setcounter{checklistnum}{0} \normalfont\Large\bfseries}{}{0em}{}[{\titlerule[0.8pt]}]
\titleformat{\subsection} {\setcounter{checklistnum}{0} \normalfont\large}{}{0em}{}[{\titlerule[0.6pt]}]
% end set section title styles -- }}}

% begin checklist symbols definition -- {{{
\usepackage{enumitem,amssymb}
\newcounter{checklistnum}
\setcounter{checklistnum}{0}
\DeclareRobustCommand{\checklistnumber}{\refstepcounter{checklistnum}\thechecklistnum}
\newlist{checklist}{itemize}{6}
\setlist[checklist,1]{
label={\color{gray}\checklistnumber}\hspace{2em}$\square$,
leftmargin=0em,
itemindent=2em
}
\setlist[checklist,2]{
label={\color{gray}\checklistnumber}\hspace{4em}$\square$,
leftmargin=0em,
itemindent=4em
}
\setlist[checklist,3]{
label={\color{gray}\checklistnumber}\hspace{6em}$\square$,
leftmargin=0em,
itemindent=6em
}
\setlist[checklist,4]{
label={\color{gray}\checklistnumber}\hspace{8em}$\square$,
leftmargin=0em,
itemindent=8em
}
\setlist[checklist,5]{
label={\color{gray}\checklistnumber}\hspace{10em}$\square$,
leftmargin=0em,
itemindent=10em
}
\setlist[checklist,6]{
label={\color{gray}\checklistnumber}\hspace{12em}$\square$,
leftmargin=0em,
itemindent=12em
}
% end checklist symbols definition -- }}}

% begin personnel macro -- {{{
\newcommand{\operator}[4]{%
  \expandafter\newcommand\csname #1\endcsname{{\color{#2}\textbf{#3}}\phantom{}}
  \expandafter\newcommand\csname #1full\endcsname{{\color{#2}\textbf{#4 [#3]}}\phantom{}}
}
% end personnel macro -- }}}

% begin flag macro -- {{{
\newcommand\flag[1]{
    \hfill \makebox(0,0){\hspace{1.7in}\textcolor{#1}{\rule{0.3in}{0.5in}}}
}
% end flag macro -- }}}

\pagenumbering{gobble}


\title{
\Huge Standard Operations Template\\
\vspace{1cm}
\Large For Tests, Internal SOPs, or Competition Procedures}


\begin{document}



\input{common/standard_titlepage}

\section{Big Title at the Start of the Document}

% begin operators declarations -- {{{
% syntax for operator macro:
% arg1: their name (so here you use \auth to insert the AUTHOR name)
% arg2: the color you want them highlighted in
% arg3: their abbreviated title (what you should use in the checklists)
% arg4: their full title (insert into document with arg1 + full, so for auth
%       it'd be "\authfull" to insert their full title). Use this in Personnel
%       Required subsection, and in Sign Off subsection.
\operator{auth}{blue}{AUTHOR}{Person Writing a Procedure}
\operator{edi}{red}{EDITOR}{People Editing the Procedure}
\operator{perf}{olive}{PERFORMER}{Person Performing the Procedure}
% end operators declarations -- }}}

% begin contents subsection -- {{{
\subsection{Contents}
This document contains the following:
\begin{itemize}
    \item A section about what people you'll need to perform this boilerplate procedure
    \item A section on how to modify this latex file to create a new procedure document
\end{itemize}
% end contents subsection -- }}}

% begin personnel required section -- {{{
\subsection{Personnel Required}
The test operations team consists of seven personnel:
\begin{checklist}
    \item The \authfull{} writes a new procedure
    \item The \edifull{} reads this new procedure and approves it before it is used
    \item The \perffull{} Performs the actions laid out in the new procedure
\end{checklist}
\setcounter{checklistnum}{0}
% end personnel required section -- }}}

% begin sign off -- {{{
\subsection{Sign-Off}
\textit{To be completed by all test personnel after reading and familiarization with procedures}
\begin{checklist}
    \item \authfull      \tabto{25em}\rule{10em}{0.4pt}\hspace{5em}\rule{10em}{0.4pt}
    \item \edifull      \tabto{25em}\rule{10em}{0.4pt}\hspace{5em}\rule{10em}{0.4pt}
    \item \perffull      \tabto{25em}\rule{10em}{0.4pt}\hspace{5em}\rule{10em}{0.4pt}
\end{checklist}
\setcounter{checklistnum}{0}
% end sign off -- }}}

% begin prior to start section -- {{{
\subsection{Prior to Start}
\begin{checklist}
    \item Items to be completed before beginning procedure
    \begin{checklist}
        \item Make sure the \auth{} has a computer they can compile \LaTeX{} on, or has access to a website that can do so
        \item Make sure the \edi{} has been asked to review a new procedure (or do this after, I'm not a cop, do what you want)
        \item Make sure caffeine is available
    \end{checklist}
\end{checklist}
\setcounter{checklistnum}{0}
% end prior to start section -- }}}

\newpage
% begin first checklist-- {{{
\subsection{How to Make a New Procedure}
% syntax for the flag macro:
% arg1: The colour you want the flag to be
% this should be useful for abort procedures that we need to put sticky notes on for irec judges to happy for
% It prints the flag all the way over the right edge so the flag should be visible even in the middle of a stack of ops
\flag{blue}
\begin{checklist}
    \item \auth{} Copy this file (standard\_template.tex) to a new latex file (extension .tex)
    \item \auth{} Change the Document Title and Section title (which will appear at the top of the first page).
    \item \auth{} Create the operators that you'll need for your procedure (use the \\operators macro, it's super helpful)
    \item \auth{} Add those operators to the Personnel Required and Sign-Off sections
    \item \auth{} Write your checklists (this part is important, which is why there's a big red square on the right margin, so you can put one of those sticky note things there and flip to it easily). \flag{red}
    \item \auth{} Send to \edi{} for review
    \item \edi{} Edit the document. Compile it, commit it, whatever. Send to \perf{} so they can do the thing.
    \item \perf{} Do the thing in the procedure.
\end{checklist}
\setcounter{checklistnum}{0}
% end first checklist -- }}}

\end{document}

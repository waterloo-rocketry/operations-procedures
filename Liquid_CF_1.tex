\documentclass[letter]{article}

%% margins and fonts
\usepackage[margin=1in]{geometry}
\renewcommand{\familydefault}{\sfdefault}

% begin package imports -- {{{
\usepackage{amsmath}
\usepackage{graphicx}
\usepackage[dvipsnames]{xcolor}
\usepackage{tabto}
\usepackage{titling}
\usepackage[iso,german]{isodate}
% end package imports -- }}}

% begin set section title styles -- {{{
\usepackage{titlesec}
\titleformat{\section} {\setcounter{checklistnum}{0} \normalfont\Large\bfseries}{}{0em}{}[{\titlerule[0.8pt]}]
\titleformat{\subsection} {\setcounter{checklistnum}{0} \normalfont\large}{}{0em}{}[{\titlerule[0.6pt]}]
% end set section title styles -- }}}

% begin checklist symbols definition -- {{{
\usepackage{enumitem,amssymb}
\newcounter{checklistnum}
\setcounter{checklistnum}{0}
\DeclareRobustCommand{\checklistnumber}{\refstepcounter{checklistnum}\thechecklistnum}
\newlist{checklist}{itemize}{6}
\setlist[checklist,1]{
label={\color{gray}\checklistnumber}\hspace{2em}$\square$,
leftmargin=0em,
itemindent=2em
}
\setlist[checklist,2]{
label={\color{gray}\checklistnumber}\hspace{4em}$\square$,
leftmargin=0em,
itemindent=4em
}
\setlist[checklist,3]{
label={\color{gray}\checklistnumber}\hspace{6em}$\square$,
leftmargin=0em,
itemindent=6em
}
\setlist[checklist,4]{
label={\color{gray}\checklistnumber}\hspace{8em}$\square$,
leftmargin=0em,
itemindent=8em
}
\setlist[checklist,5]{
label={\color{gray}\checklistnumber}\hspace{10em}$\square$,
leftmargin=0em,
itemindent=10em
}
\setlist[checklist,6]{
label={\color{gray}\checklistnumber}\hspace{12em}$\square$,
leftmargin=0em,
itemindent=12em
}
% end checklist symbols definition -- }}}

% begin personnel macro -- {{{
\newcommand{\operator}[4]{%
  \expandafter\newcommand\csname #1\endcsname{{\color{#2}\textbf{#3}}\phantom{}}
  \expandafter\newcommand\csname #1full\endcsname{{\color{#2}\textbf{#4 [#3]}}\phantom{}}
}
% end personnel macro -- }}}

% begin flag macro -- {{{
\newcommand\flag[1]{
    \hfill \makebox(0,0){\hspace{1.7in}\textcolor{#1}{\rule{0.3in}{0.5in}}}
}
% end flag macro -- }}}

\pagenumbering{gobble}


\title{
\Huge Unnamed Liquid Rocket Engine\\
Cold Flow 1\\
\vspace{1cm}
\Large Cold Flow Test Operations Procedures}

%begin operators definitions -- {{{
\operator{ops}{Blue}{OPS}{Operations Director}
\operator{primary}{Fuchsia}{PRIMARY}{Primary Fill Operator}
\operator{secondary}{Emerald}{SECONDARY}{Secondary Fill Operator}
\operator{daq}{OliveGreen}{DAQ}{DAQ Technician}
\operator{control}{Maroon}{CONTROL}{Control System Operator}
\operator{peri}{red}{P1}{Perimeter Guard 1}
\operator{perii}{red}{P2}{Perimeter Guard 2}
\operator{periii}{red}{P3}{Perimeter Guard 3}
%end operators definitions -- }}}}}

\begin{document}
%%%%%%%%%%%%%%%%%%%%%%%%%%%%%%%%%%%%%%%%%%%%%%%%%%%%%%%%%%%%%%%%%%%%%%%%%%%%%%%%%
% Waterloo Rocketry Standard template for operations procedures                 %
% Copyright claimed by Aaron Morrison (akmorris@uwaterloo.ca)                   %
% Any commercial entity wishing to use this template in any way must provide    %
% to Waterloo Rocketry, one two-four (24 pack) of a non-shitty                  %
% type of beer or an amount of Canadian dollars roughly equalling               %
% the current cost of such a purchase. If you do so, then use it                %
% however you want, and I claim in no way that it compiles properly,            %
% and I take no responsibilty for any loss you incurr from using                %
% this template, and make no promises to support you in any technical           %
% capacity for sustained use.                                                   %
%%%%%%%%%%%%%%%%%%%%%%%%%%%%%%%%%%%%%%%%%%%%%%%%%%%%%%%%%%%%%%%%%%%%%%%%%%%%%%%%%

% begin titlepage -- {{{
\begin{center}
\vspace*{7cm}
\hspace{7em}\includegraphics[width=30em]{common/mono_horizontal_standard}
\newline
\rule{50em}{2pt}

\vspace{1cm}
\thetitle

\vspace*{\fill}
Compiled on \today
\end{center}
\newpage
% end titlepage -- }}}


\section{Cold Flow Test Operations Procedures}

% begin contents subsection -- {{{
\subsection{Contents}
This document contains the following procedure:
\begin{itemize}
    \item The \textit{Cold Flow Test} procedure comprises steps for conducting a cold flow test of the engine and fill system using the electrical control system and motorized ball valves.
\end{itemize}
% end contents subsection -- }}}

% begin personnel required section -- {{{
\subsection{Personnel Required}
The test operations team consists of nine personnel:
\begin{checklist}
    \item The \opsfull{} directs operations procedures and communicates with the other test personnel.
    \item The \primaryfull{} is the main system operator. \primary{} operates all manual valves as well as the test control system.
    \item The \secondaryfull{} is the backup for \primary{}, and communicates with OPS. If \primary{} becomes incapacitated, \secondary{} is responsible for removing them from danger.
    \item The \daqfull{} monitors and operates the test data acquisition system.
    \item The \controlfull{} operates the test control system, including actuation of remote valves.
    \item \perifull{}, \periifull{}, and \periiifull{} ensure that no unauthorized personnel enter the testing area during test operations.
\end{checklist}
\setcounter{checklistnum}{0}
% end personnel required section -- }}}

% begin sign off -- {{{
\subsection{Sign-Off}
\textit{To be completed by all test personnel after reading and familiarization with procedures}
\begin{checklist}
    \item \opsfull      \tabto{25em}\rule{10em}{0.4pt}\hspace{5em}\rule{10em}{0.4pt}
    \item \primaryfull  \tabto{25em}\rule{10em}{0.4pt}\hspace{5em}\rule{10em}{0.4pt}
    \item \secondaryfull\tabto{25em}\rule{10em}{0.4pt}\hspace{5em}\rule{10em}{0.4pt}
    \item \daqfull      \tabto{25em}\rule{10em}{0.4pt}\hspace{5em}\rule{10em}{0.4pt}
    \item \controlfull      \tabto{25em}\rule{10em}{0.4pt}\hspace{5em}\rule{10em}{0.4pt}
    \item \perifull     \tabto{25em}\rule{10em}{0.4pt}\hspace{5em}\rule{10em}{0.4pt}
    \item \periifull    \tabto{25em}\rule{10em}{0.4pt}\hspace{5em}\rule{10em}{0.4pt}
    \item \periiifull    \tabto{25em}\rule{10em}{0.4pt}\hspace{5em}\rule{10em}{0.4pt}
\end{checklist}
\setcounter{checklistnum}{0}
% end sign off -- }}}

\newpage
% begin prior to start section -- {{{
\subsection{Prior to Start}
\begin{checklist}
    \item Ensure that the following procedures are complete:
    \begin{checklist}
        \item Oxidizer Tank Assembly procedure
        \item Plumbing Setup procedure
        \item Oxidizer Tank Stand Setup procedure
        \item Tank Heating Setup procedure
        \item Test Stand Setup procedure
        \item Data Acquisition Setup procedure
        \item Test Control System Setup procedure
    \end{checklist}
    \item Ensure that all technicians as defined above are available and have completed the sign-off.
    \item Ensure that all spectators and test personnel are wearing safety glasses.
    \item Ensure that \primary{} and \secondary{} are wearing face shields and have no exposed skin.
    \item Ensure that \primary{} is wearing thermal gloves.
    \item Ensure that \ops{} is in possession of the system control key.
\end{checklist}
\setcounter{checklistnum}{0}
% end prior to start section -- }}}

\newpage
% begin Cold Flow Test Procedure section -- {{{
\subsection{Cold Flow Test Procedure}
\begin{checklist}
    \item \primary{}: Confirm that the following valves are initially closed:
    \begin{checklist}
		\item BA-1
        \item BA-3
        \item BA-5
        \item BA-6
		\item BA-9
		\item MV-1
		\item MV-2
		\item MV-3
    \end{checklist}
    \item \primary{}: Confirm that the following valves are initially open:
    \begin{checklist}
  		\item BA-2
		\item BA-4
    \end{checklist}
    \item \daq{}: Confirm that all pressure transducers are reading atmospheric pressure.
    \item \daq{}: Confirm that all load cells are reading the determined zero point.
    \item \textbf{\textit{PAUSE POINT}}
    \item \peri{}, \perii{}, and \periii{}: Close the perimeter and do not allow any further personnel to enter the testing area.
    \item \secondary: Confirm that no personnel are present in the testing area other than \primary{} and \secondary.
    \item \primary{}: Remove the cap from TK-1.
    \item \primary{}: Connect the pressurant line to TK-1, hand tighten, and then tighten with a wrench. Do not force the connection.
    \item \primary{}: Slowly open GA-1 through $\frac{3}{4}$  of a turn.
    \begin{checklist}[label=$\bullet$]
        \item If leaks are observed:
        \begin{checklist}
            \item \primary{}: Close GA-1.
            \item \primary{}: TODO
        \end{checklist}
    \end{checklist}
    \item \primary{}: Adjust CV-4 to 360 psi.
    \item \daq{}: Communicate the pressure reading of PT-1.
    \item \ops{}: Record the pressure reading of PT-1.
    \item \primary{}: Communicate the pressure readings of PI-3 and PI-4.
    \item \ops{}: Record the pressure readings of PI-3 and PI-4.
    \item \primary{} and \secondary{}: Retreat back to Mission Control.
    \item \control{}: Perform the following control system checks:
    \begin{checklist}
        \item Confirm that all actuator controls are in the ``closed'' position:
        \begin{checklist}
            \item Remote Fill Valve
            \item Motorized Vent Valve
            \item Pressurant Valve
        \end{checklist}
    \end{checklist}
    \item \textbf{\textit{PAUSE POINT}}
    \item \ops{}: Poll the following personnel for GO/NO GO status:
    \begin{checklist}
		\item \control{}
        \item \daq{}
        \item \primary{}
        \item \secondary{}
		\item \peri{}
		\item \perii{}
		\item \periii{}
    \end{checklist}
    \item \ops{}: Give the system control key to \control{}.
    \item \control{}: Engage the key switch and power on the control boxes.
    \item \control{}: Open the Pressurant Valve.
    \item \daq{}: Monitor PT-1 and the fuel tank load cell during fuel pressurization.
    \item \daq{}: Proceed when the fuel tank mass is stable.
    \item \textbf{\textit{PAUSE POINT}}
    \item \primary{}: Conduct the cold flow test by opening BA-9 using the ropes.
    \item \ops{}: Proceed when water flow has ceased or after 15 seconds have elapsed.
    \item \control{}: Close the Pressurant Valve.
    \item \textbf{\textit{PAUSE POINT}}
    \item \daq{}: Confirm that PT-1 is reading atmospheric pressure.
	\item \control{}: Disengage the key switch and disable actuators.    
    \item \primary{} and \secondary{}: Approach the test plumbing.
    \item \primary{}: Close BA-2.
    \item \primary{}: Open BA-1.
    \item \primary{}: Adjust CV-4 to 600 psi.
    \item \primary{}: Disconnect the pressure relief valve assembly from the fuel plumbing and connect it to the oxidizer plumbing.
    \item \daq{}: Communicate the pressure reading of PT-1.
    \item \ops{}: Record the pressure reading of PT-1.
    \item \primary{}: Communicate the pressure readings of PI-3 and PI-4.
    \item \ops{}: Record the pressure readings of PI-3 and PI-4.
    \item \primary{}: Remove the cap from SC-1.
    \item \primary{}: Connect the fill line to SC-1, hand tighten, and then tighten with a wrench. Do not force the connection.
    \item \primary{}: Slowly open the Cylinder Valve through $\frac{3}{4}$  of a turn.
    \begin{checklist}[label=$\bullet$]
        \item If leaks are observed:
        \begin{checklist}
            \item \primary{}: Close the Cylinder Valve.
            \item \primary{}: Slowly open BA-3.
            \item \primary{}: Slowly open BA-5.
            \item \daq{}: Confirm that PT-2 is reading atmospheric pressure.
            \item \ops{}: Abort test procedures and revisit plumbing setup.
        \end{checklist}
    \end{checklist}
    \item \primary{}: Communicate the reading of PI-2.
    \item \daq{}: Communicate the reading of PT-2.
    \item \daq{}: Confirm that the two pressure measurements are in agreement.
    \item \primary{} and \secondary{}: Retreat back to Mission Control.
    \item \control{}: Perform the following control system checks:
    \begin{checklist}
        \item Confirm that all actuator controls are in the ``closed'' position:
        \begin{checklist}
            \item Remote Fill Valve
            \item Motorized Vent Valve
            \item Pressurant Valve
        \end{checklist}
    \end{checklist}
    \item \textbf{\textit{PAUSE POINT}}
    \item \ops: Give the system control key to \control{}.
    \item \control{}: Engage the key switch and power on the control boxes.
    \item \control{}: Open the Motorized Vent Valve.
    \item \control{}: Open the Remote Fill Valve.
    \begin{checklist}[label=$\bullet$]
        \item If leaks are observed:
        \begin{checklist}
            \item \control{}: Close the Remote Fill Valve.
            \item \primary{}: Open BA-6 using the ropes.
            \item \ops: Proceed only when the oxidizer tank has fully vented.
            \item \primary{}: Close the Cylinder Valve.
            \item \daq{}: Confirm that the Fill Pressure Transducer is reading atmospheric pressure.
            \item \ops{}: Abort test procedures and revisit plumbing setup.
        \end{checklist}
    \end{checklist}
    \item \ops{}: Proceed only when a white plume is visible from the vent plug.
    \item \control{}: Close the Motorized Vent Valve.
    \item \control{}: Close the Remote Fill Valve.
    \item \control{}: Open the Pressurant Valve.
    \item \daq{}: Proceed when PT-1 is stable at 600 psi.
    \item \textbf{\textit{PAUSE POINT}}
    \item \primary{}: Conduct the cold flow test by opening BA-6 using the ropes.
    \item \ops{}: Proceed when carbon dioxide flow has ceased or after 15 seconds have elapsed.
    \item \control{}: Close the Pressurant Valve.
    \item \textbf{\textit{PAUSE POINT}}
    \item \ops{}: Proceed only when the oxidizer tank has fully vented.
    \item \daq{}: Confirm that PT-1 is reading atmospheric pressure.
    \item \control{}: Open the Motorized Vent Valve.
    \item \primary{} and \secondary{}: Approach the test plumbing.
    \item \primary{}: Close the Cylinder Valve.
    \item \primary{}: Slowly open BA-3.
    \item \control{}: Open the Remote Fill Valve.
    \item \primary{}: Close GA-1.
    \item \control{}: Open MV-2.
    \item \primary{}: Slowly open BA-2.
    \item \primary{}: Slowly open BA-5.
    \item \primary{}: Disconnect the fill line from SC-1.
    \item \primary{}: Replace the cap on SC-1.
    \item \daq{}: Confirm that all pressure transducers are reading atmospheric pressure.
    \item \peri{}, \perii{}, and \periii{}: Open the perimeter.
    \item \ops{}: Proceed with teardown and disassembly.
\end{checklist}
% end Cold Flow Test Procedure section -- }}}

\end{document}
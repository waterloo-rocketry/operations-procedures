\documentclass[letter]{article}

%% margins and fonts
\usepackage[margin=1in]{geometry}
\renewcommand{\familydefault}{\sfdefault}

% begin package imports -- {{{
\usepackage{amsmath}
\usepackage{graphicx}
\usepackage[dvipsnames]{xcolor}
\usepackage{tabto}
\usepackage{titling}
\usepackage[iso,german]{isodate}
% end package imports -- }}}

% begin set section title styles -- {{{
\usepackage{titlesec}
\titleformat{\section} {\setcounter{checklistnum}{0} \normalfont\Large\bfseries}{}{0em}{}[{\titlerule[0.8pt]}]
\titleformat{\subsection} {\setcounter{checklistnum}{0} \normalfont\large}{}{0em}{}[{\titlerule[0.6pt]}]
% end set section title styles -- }}}

% begin checklist symbols definition -- {{{
\usepackage{enumitem,amssymb}
\newcounter{checklistnum}
\setcounter{checklistnum}{0}
\DeclareRobustCommand{\checklistnumber}{\refstepcounter{checklistnum}\thechecklistnum}
\newlist{checklist}{itemize}{6}
\setlist[checklist,1]{
label={\color{gray}\checklistnumber}\hspace{2em}$\square$,
leftmargin=0em,
itemindent=2em
}
\setlist[checklist,2]{
label={\color{gray}\checklistnumber}\hspace{4em}$\square$,
leftmargin=0em,
itemindent=4em
}
\setlist[checklist,3]{
label={\color{gray}\checklistnumber}\hspace{6em}$\square$,
leftmargin=0em,
itemindent=6em
}
\setlist[checklist,4]{
label={\color{gray}\checklistnumber}\hspace{8em}$\square$,
leftmargin=0em,
itemindent=8em
}
\setlist[checklist,5]{
label={\color{gray}\checklistnumber}\hspace{10em}$\square$,
leftmargin=0em,
itemindent=10em
}
\setlist[checklist,6]{
label={\color{gray}\checklistnumber}\hspace{12em}$\square$,
leftmargin=0em,
itemindent=12em
}
% end checklist symbols definition -- }}}

% begin personnel macro -- {{{
\newcommand{\operator}[4]{%
  \expandafter\newcommand\csname #1\endcsname{{\color{#2}\textbf{#3}}\phantom{}}
  \expandafter\newcommand\csname #1full\endcsname{{\color{#2}\textbf{#4 [#3]}}\phantom{}}
}
% end personnel macro -- }}}

% begin flag macro -- {{{
\newcommand\flag[1]{
    \hfill \makebox(0,0){\hspace{1.7in}\textcolor{#1}{\rule{0.3in}{0.5in}}}
}
% end flag macro -- }}}

\pagenumbering{gobble}


\title{
\Huge Unexploded Ordnance Hybrid Rocket\\
2018 IREC\\
\vspace{1cm}
\Large Launch Operations Procedures}

%begin operators definitions -- {{{
\operator{ops}{Blue}{OPS}{Operations Director}
\operator{control}{Maroon}{CONTROL}{Control System Operator}
\operator{primary}{Fuchsia}{PRIMARY}{Primary Fill Operator}
\operator{secondary}{Emerald}{SECONDARY}{Secondary Fill Operator}
%end operators definitions -- }}}}}

\begin{document}
\input{common/standard_titlepage}

\section{Background and Reference}

% begin contents subsection -- {{{
\subsection{Contents}
This document contains two nominal procedures:
\begin{itemize}
    \item \textbf{N1}, \textit{Final Setup and Pre-Launch Checks}, comprises the final checks and tests performed on the Remote Launch Control System (RLCS) prior to rocket launch, as well as avionics systems arming.
    \item \textbf{N2}, \textit{Fill and Launch Operations}, comprises steps for oxidizer fill and rocket launch.
\end{itemize}
Additionally, this document contains five abort procedures:
\begin{itemize}
    \item \textbf{A1}, \textit{Abort Procedure - Leak At Supply Plumbing}, is used if a plumbing leak is detected when the supply cylinder is initially opened.
    \item \textbf{A2}, \textit{Abort Procedure - Low Supply Pressure}, is used if the oxidizer pressure is below the acceptable limit for launch.
    \item \textbf{A3}, \textit{Abort Procedure - High Supply Pressure}, is used if the oxidizer pressure is above the acceptable limit for launch.
    \item \textbf{A4}, \textit{Abort Procedure - Leak At Fill Plumbing}, is used if a plumbing leak is detected during manual fill leak checks.
    \item \textbf{A5}, \textit{Abort Procedure - Remote Disconnect or Ignition Failure}, is used if the remote disconnect or ignition systems fail, necessitating a full vent of the oxidizer tank.
\end{itemize}
% end contents subsection -- }}}

% begin personnel required section -- {{{
\subsection{Personnel Required}
The launch operations team consists of four personnel:
\begin{checklist}
    \item The \opsfull{} is stationed at Launch Control. \ops{} directs operations procedures and communicates with the other launch personnel.
    \item The \controlfull{} is stationed at Launch Control and is responsible for operation of RLCS, including remote fill, disconnect, and ignition.
    \item The \primaryfull{} is initially stationed at the Launch Tower and carries out all tasks occurring at the Launch Tower. \primary{} engages the remote disconnect system, arms the vehicle recovery deployment system, connects the ignition wires to the rocket, and operates all manual valves during the manual portion of fill.
    \item The \secondaryfull{} is the backup for \primary{}, and communicates with \ops{}. If \primary{} becomes incapacitated, \secondary{} is responsible for removing them from danger.
\end{checklist}
\setcounter{checklistnum}{0}
% end personnel required section -- }}}

% begin sign off -- {{{
\subsection{Sign-Off}
\textit{To be completed by all test personnel after reading and familiarization with procedures}
\begin{checklist}
    \item \opsfull      \tabto{25em}\rule{10em}{0.4pt}\hspace{5em}\rule{10em}{0.4pt}
    \item \controlfull      \tabto{25em}\rule{10em}{0.4pt}\hspace{5em}\rule{10em}{0.4pt}
    \item \primaryfull  \tabto{25em}\rule{10em}{0.4pt}\hspace{5em}\rule{10em}{0.4pt}
    \item \secondaryfull\tabto{25em}\rule{10em}{0.4pt}\hspace{5em}\rule{10em}{0.4pt}
\end{checklist}
\setcounter{checklistnum}{0}
% end sign off -- }}}
\newpage{}

\section{[N1] Final Setup and Pre-Launch Checks} \flag{Blue}
% begin prior to start section -- {{{
\subsection{Prior to Start}
\begin{checklist}
    \item Ensure that the following procedures are complete:
    \begin{checklist}
        \item Rocket Assembly procedure
        \item RLCS Setup procedure
        \item Launch Tower Setup procedure
    \end{checklist}
    \item Ensure that all personnel as defined above are available and have completed the sign-off.
    \item Ensure that the following personnel have walkie-talkies and communication is functional:
    \begin{checklist}
        \item \ops{}
        \item \control{}
        \item \primary{}
        \item \secondary{}
    \end{checklist}
    \item Ensure that \ops{} is in possession of the system control key.
    \item Ensure that the locations of Launch Control, Launch Tower, and the Minimum Safe Distance are clearly defined:
    
    \begin{center}
    \begin{tabular}{|l|c|r|}
    \hline{}
    Launch Control&Launch Tower&Minimum Safe Distance\\ \hline{}
    & & \\ \hline{}
    \end{tabular}
    \end{center}
    
\end{checklist}
\setcounter{checklistnum}{0}
% end prior to start section -- }}}

% begin Fill System Check section -- {{{
\subsection{Nominal Procedure}
\begin{checklist}
    \item \primary{}: Confirm that the following valves are initially closed:
\end{checklist}
\setcounter{checklistnum}{0}
% end Fill System Check section -- }}}
\newpage

\section{[N2] Fill and Launch Operations} \flag{brown}
% begin prior to start section -- {{{
\subsection{Prior to Start}
\begin{checklist}
    \item Ensure that the following procedure is complete:
    \begin{checklist}
        \item \textbf{N1}, \textit{Final Setup and Pre-Launch Checks}
    \end{checklist}
    \item Ensure that all personnel are available and have completed the sign-off.
    \item Ensure that the following personnel have walkie-talkies and communication is functional:
    \begin{checklist}
        \item \ops{}
        \item \control{}
        \item \primary{}
        \item \secondary{}
    \end{checklist}
    \item Ensure that \primary{} and \secondary{} are wearing face shields and have no exposed skin.
    \item Ensure that \primary{} is wearing thermal gloves.
    \item Ensure that \ops{} is in possession of the system control key.
\end{checklist}
\setcounter{checklistnum}{0}
% end prior to start section -- }}}

% begin Remote Control Procedure section -- {{{
\subsection{Nominal Procedure}
\begin{checklist}
    \item \secondary: Confirm that no personnel other than \primary{} and \secondary{} are within the Minimum Safe Distance.
    \item \ops: Confirm that the actuator key switch is disabled and that only \ops{} is in possession of the system control  key.
    \item \ops: Confirm that the Range Safety Officer and Launch Control Officer have given clearance to proceed with fill procedures.
    \item \control: Confirm that the RLCS client-side box is on and displaying DAQ information.
    \item \primary{}: Confirm that the following valves are initially closed:
    \begin{checklist}
        \item Cylinder Valve
        \item Remote Fill Valve
        \item Parallel Fill Valve
        \item Series Fill Valve
        \item Line Vent Valve
        \item Parallel Vent Valve
    \end{checklist}
    \item \ops{}: Confirm that the Tank Vent Valve is initially open.
    \item \ops{}: Confirm that the Pressure Relief Valve is initially closed.
    \item \ops{}: Confirm that the Injector Valve is initially closed.
    \item \primary: Slowly open the Cylinder Valve through $\frac{3}{4}$ of a turn.
    \begin{checklist}[label=$\bullet$]
        \item If leaks are observed:
        \begin{checklist}
            \item \ops{}: Proceed to procedure A1.
        \end{checklist}
    \end{checklist}
    \item \primary{}: Communicate the supply line pressure as visible on the Pressure Gauge.
    \begin{checklist}[label=$\bullet$]
        \item If the supply line pressure is below 800 psi:
        \begin{checklist}
            \item \ops{}: Proceed to procedure \textbf{A2}.
        \end{checklist}
        \item If the supply line pressure exceeds 1050 psi:
        \begin{checklist}
            \item \ops{}: Proceed to procedure \textbf{A3}.
        \end{checklist}
    \end{checklist}
    \item \control: Confirm that the supply line pressure as read by \primary{} agrees with the supply line pressure measured by the DAQ system.
    \item \primary: Slowly open the Parallel Fill Valve.
    \begin{checklist}[label=$\bullet$]
        \item If leaks are observed:
        \begin{checklist}
            \item \ops{}: Proceed to procedure \textbf{A4}.
        \end{checklist}
    \end{checklist}
    \item \control: Confirm that the pressures in the fill lines and in the oxidizer tank are increasing.
    \item \primary: Close the Parallel Fill Valve.
    \item \primary: Open the Series Fill Valve.
    \item \primary{} and \secondary {}: Retreat to the Minimum Safe Distance.
    \item \secondary{}: Confirm that \primary{} and \secondary{} are at the Minimum Safe Distance.
    \item \textbf{\textit{PAUSE POINT}}
    \item \ops: Give the system control key to \control{}.
    \item \control: Confirm that all actuator controls are in the off state:
    \begin{checklist}
        \item Remote Fill Valve
        \item Line Vent Valve
        \item Remote Disconnect
        \item Tank Vent Valve
        \item Primary Ignition
        \item Secondary Ignition
        \item Injector Valve
    \end{checklist}
    \item \control{}: Engage the key switch and enable actuators.
    \item \control{}: Open the Remote Fill Valve.
    \item \control{}: Monitor the RLCS display for rocket mass and oxidizer tank pressure.
    \item \ops{}: Proceed only when the following is true:
    \begin{checklist}
        \item Rocket mass plateaus
        \item Oxidizer tank pressure is within the acceptable limits
    \end{checklist}
    \item \control{}: Close the Tank Vent Valve.
    \item \control{}: Close the Remote Fill Valve.
    \item \control{}: Open the Remote Vent Valve.
    \item \control{}: Confirm that the fill line pressure is atmospheric.
    \item \control{}: Actuate Remote Disconnect.
     \begin{checklist}[label=$\bullet$]
        \item If Remote Disconnect fails to actuate:
        \begin{checklist}
            \item \ops{}: Proceed to procedure \textbf{A5}.
        \end{checklist}
    \end{checklist}
    \item \textbf{\textit{PAUSE POINT}}
    \item \ops: Perform pre-launch checks:
    \begin{checklist}
        \item Request clearance for launch from the Launch Control Officer.
        \item Confirm that all members are aware of launch.
    \end{checklist}    
    \item \primary{}: Perform engine startup procedure:
    \begin{checklist}
        \item Arm the Primary Ignition switch.
        \item Hold down the Fire button until the Primary current reading drops to 0 A.
        \begin{checklist}[label=$\bullet$]
            \item In the event of a failed ignition (current drop not observed within 1 minute):
            \begin{checklist}
                \item \primary: Disarm the Primary Ignition switch.
                \item \primary: Arm the Secondary Ignition switch.
                \item \ops: Revisit ignition procedure.
            \end{checklist}
            \item In the event of a second failed ignition (current drop not observed within 1 minute):
            \begin{checklist}
                \item \primary: Disarm the Secondary Ignition switch.
                \item \ops: Proceed to procedure \textbf{A5}.
            \end{checklist}
        \end{checklist}
        \item \primary: Start the engine by opening the Injector Valve.
    \end{checklist}
    \item \textbf{ALL}: Observe the rocket during takeoff, ascent, and recovery:
    \begin{checklist}
        \item First vehicle motion
        \item Launch rail departure
        \item Engine burnout
        \item Payload deployment
        \item Drogue parachute deployment
        \item Main parachute deployment
        \item Approximate recovery area/direction
    \end{checklist}
    \item \control{}: Disarm RLCS:
    \begin{checklist}
        \item Disable actuator control by removing the system control key.
        \item Give the system control  key to \ops{}.
    \end{checklist}
    \item \ops: Confirm that RLCS is disarmed and \ops{} is in possession of the system control  key.
    \item \ops{}: Proceed only when clearance is received from the Launch Control Officer to approach the Launch Tower.
    \item \primary{} and \secondary: Approach the Launch Tower.
    \item \primary{}: Close the Cylinder Valve.
    \item \primary{}: Open the Parallel Vent Valve.
    \item \primary{}: Slowly open the Parallel Fill Valve.
    \item \primary{} and \secondary: Retreat 20 ft from the fill system.
    \item \ops: Give the master key to \control{}
    \item \control{}: Engage the key switch and enable actuators.
    \item \control{}: Open the Remote Fill Valve.
    \item \control{}: Confirm that the supply line pressure is atmospheric.
    \item \primary{}: Disconnect the fill line from the supply cylinder.
    \item \primary{}: Replace the cap on the nitrous oxide supply cylinder.
    \item \ops{}: Proceed with teardown and disassembly.
\end{checklist}
\setcounter{checklistnum}{0}
% end Remote Control Procedure section -- }}}

\newpage
    
\section{Abort Procedures} \flag{red}
% begin A1 section -- {{{
\subsection{[A1] Abort Procedure - Leak At Supply Plumbing}
\begin{checklist}
    \item \primary{}: Close the Cylinder Valve.
    \item \primary{}: Slowly open the Parallel Fill Valve.
    \item \primary{}: Slowly open the Parallel Vent Valve.
    \item \control{}: Confirm that the fill and supply pressures are atmospheric.
    \item \primary{}: Disarm the system:
        \begin{checklist}    
            \item Disconnect the ignition leads from the rocket.
            \item Detatch the torsion springs from the disconnect mechanism.
            \item Disarm the recovery electronics system using the magnetic switches.
            \item Disarm the payload using the transponder.
            \item Disconnect the fill line from the supply cylinder.
            \item Replace the cap on the nitrous oxide supply cylinder.
        \end{checklist}
    \item \ops{}: Revisit plumbing setup.
\end{checklist}
\setcounter{checklistnum}{0}
% end A1 section -- }}}

% begin A2 section -- {{{
\subsection{[A2] Abort Procedure - Low Supply Pressure}
\begin{checklist}
    \item \primary{}: Close the Cylinder Valve.
    \item \primary{}: Slowly open the Parallel Fill Valve.
    \item \primary{}: Slowly open the Parallel Vent Valve.
    \item \control{}: Confirm that the fill and supply pressures are atmospheric.
    \item \primary{}: Allow the supply cylinder to warm up.
    \item \ops{}: Revisit \textbf{N1}.
\end{checklist}
\setcounter{checklistnum}{0}
% end A2 section -- }}}comprises the final checks and tests performed on the Remote Launch Control System (RLCS) prior to rocket launch, as well as avionics systems arming

% begin A3 section -- {{{
\subsection{[A3] Abort Procedure - High Supply Pressure}
\begin{checklist}
    \item \primary{}: Close the Cylinder Valve.
    \item \primary{}: Slowly open the Parallel Fill Valve.
    \item \primary{}: Slowly open the Parallel Vent Valve.
    \item \control{}: Confirm that the fill and supply pressures are atmospheric.
    \item \primary{}: Disarm the system:
        \begin{checklist}    
            \item Disconnect the ignition leads from the rocket.
            \item Detatch the torsion springs from the disconnect mechanism.
            \item Disarm the recovery electronics system using the magnetic switches.
            \item Disarm the payload using the transponder.
            \item Disconnect the fill line from the supply cylinder.
            \item Replace the cap on the nitrous oxide supply cylinder.
        \end{checklist}
    \item \ops{}: Revisit cylinder cooling methods.
\end{checklist}
\setcounter{checklistnum}{0}
% end A3 section -- }}}

% begin A4 section -- {{{
\subsection{[A4] Abort Procedure - Leak At Fill Plumbing}
\begin{checklist}
    \item \primary{}: Close the Parallel Fill Valve.
    \item \primary{}: Close the Cylinder Valve.
    \item \primary{}: Slowly open the Parallel Fill Valve.
    \item \primary{}: Slowly open the Parallel Vent Valve.
    \item \control{}: Confirm that the fill and supply pressures are atmospheric.
    \item \primary{}: Disarm the system:
        \begin{checklist}    
            \item Disconnect the ignition leads from the rocket.
            \item Detatch the torsion springs from the disconnect mechanism.
            \item Disarm the recovery electronics system using the magnetic switches.
            \item Disarm the payload using the transponder.
            \item Disconnect the fill line from the supply cylinder.
            \item Replace the cap on the nitrous oxide supply cylinder.
        \end{checklist}
    \item \ops{}: Revisit plumbing setup.
\end{checklist}
\setcounter{checklistnum}{0}
% end A4 section -- }}}

% begin A5 section -- {{{
\subsection{[A5] Abort Procedure - Remote Disconnect or Ignition Failure}
\begin{checklist}
    \item \control{}: Open the Tank Vent Valve.
    \item \control{}: Monitor the RLCS display for rocket mass and oxidizer tank pressure as the oxidizer tank vents.
    \item \ops{}: Proceed only when the following is true:
    \begin{checklist}
        \item Rocket mass is equal to the pre-launch recorded mass
        \item Oxidizer tank pressure is atmospheric
        \item The Launch Control Officer has given clearance to approach the Launch Tower.
    \end{checklist}
    \item \primary{} and \secondary: Approach the Launch Tower.
    \item \primary{}: Close the Cylinder Valve.
    \item \primary{}: Open the Parallel Vent Valve.
    \item \primary{}: Slowly open the Parallel Fill Valve.
    \item \primary{} and \secondary: Retreat 20 ft from the fill system.
    \item \ops: Give the system control key to \control{}
    \item \control{}: Engage the system control switch and enable actuators.
    \item \control{}: Open the Remote Fill Valve.
    \item \control{}: Confirm that the supply line pressure is atmospheric.
    \item \primary{}: Disarm the system:
        \begin{checklist}    
            \item Disconnect the ignition leads from the rocket.
            \item Detatch the torsion springs from the disconnect mechanism.
            \item Disarm the recovery electronics system using the magnetic switches.
            \item Disarm the payload using the transponder.
            \item Disconnect the fill line from the supply cylinder.
            \item Replace the cap on the nitrous oxide supply cylinder.
        \end{checklist}
    \item \ops{}: Proceed with teardown and disassembly.
\end{checklist}
\setcounter{checklistnum}{0}
% end A5 section -- }}}

\end{document}

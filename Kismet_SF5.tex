\documentclass[letter]{article}

%% margins and fonts
\usepackage[margin=1in]{geometry}
\renewcommand{\familydefault}{\sfdefault}

% begin package imports -- {{{
\usepackage{amsmath}
\usepackage{graphicx}
\usepackage[dvipsnames]{xcolor}
\usepackage{tabto}
\usepackage{titling}
\usepackage[iso,german]{isodate}
% end package imports -- }}}

% begin set section title styles -- {{{
\usepackage{titlesec}
\titleformat{\section} {\setcounter{checklistnum}{0} \normalfont\Large\bfseries}{}{0em}{}[{\titlerule[0.8pt]}]
\titleformat{\subsection} {\setcounter{checklistnum}{0} \normalfont\large}{}{0em}{}[{\titlerule[0.6pt]}]
% end set section title styles -- }}}

% begin checklist symbols definition -- {{{
\usepackage{enumitem,amssymb}
\newcounter{checklistnum}
\setcounter{checklistnum}{0}
\DeclareRobustCommand{\checklistnumber}{\refstepcounter{checklistnum}\thechecklistnum}
\newlist{checklist}{itemize}{6}
\setlist[checklist,1]{
label={\color{gray}\checklistnumber}\hspace{2em}$\square$,
leftmargin=0em,
itemindent=2em
}
\setlist[checklist,2]{
label={\color{gray}\checklistnumber}\hspace{4em}$\square$,
leftmargin=0em,
itemindent=4em
}
\setlist[checklist,3]{
label={\color{gray}\checklistnumber}\hspace{6em}$\square$,
leftmargin=0em,
itemindent=6em
}
\setlist[checklist,4]{
label={\color{gray}\checklistnumber}\hspace{8em}$\square$,
leftmargin=0em,
itemindent=8em
}
\setlist[checklist,5]{
label={\color{gray}\checklistnumber}\hspace{10em}$\square$,
leftmargin=0em,
itemindent=10em
}
\setlist[checklist,6]{
label={\color{gray}\checklistnumber}\hspace{12em}$\square$,
leftmargin=0em,
itemindent=12em
}
% end checklist symbols definition -- }}}

% begin personnel macro -- {{{
\newcommand{\operator}[4]{%
  \expandafter\newcommand\csname #1\endcsname{{\color{#2}\textbf{#3}}\phantom{}}
  \expandafter\newcommand\csname #1full\endcsname{{\color{#2}\textbf{#4 [#3]}}\phantom{}}
}
% end personnel macro -- }}}

% begin flag macro -- {{{
\newcommand\flag[1]{
    \hfill \makebox(0,0){\hspace{1.7in}\textcolor{#1}{\rule{0.3in}{0.5in}}}
}
% end flag macro -- }}}

\pagenumbering{gobble}


\title{\Huge Kismet Hybrid Rocket Engine\\
Static Fire 5\\
\vspace{1cm}
\Large Static Fire Test Operations Procedures}

%begin operators definitions -- {{{
\operator{ops}{Blue}{OPS}{Operations Director}
\operator{primary}{Fuchsia}{PRIMARY}{Primary Fill Operator}
\operator{secondary}{Emerald}{SECONDARY}{Secondary Fill Operator}
\operator{daq}{OliveGreen}{DAQ}{DAQ Technician}
\operator{control}{Maroon}{CONTROL}{Control System Operator}
\operator{peri}{red}{P1}{Perimeter Guard 1}
\operator{perii}{red}{P2}{Perimeter Guard 2}
\operator{periii}{red}{P3}{Perimeter Guard 3}
\operator{periv}{Red}{P4}{Perimeter Guard 4}
%end operators definitions -- }}}}}

\begin{document}
\input{common/standard_titlepage}

\section{Static Fire Test Operations Procedures}

%begin contents subsection -- {{{
\subsection{Contents}
This document contains two procedures:
\begin{itemize}
	\item The \textit{Fill System Check} procedure comprises steps for validating the integrity of the system plumbing and correct operation of the test data acquisition system, using carbon dioxide.
	\item The \textit{Static Fire Test} comprises steps for operating the fill system and conducting a static fire test of the engine. 
\end{itemize}

% begin personnel required section -- {{{
\subsection{Personnel Required}
The test operations team consists of nine personnel:
\begin{checklist}
    \item The \opsfull{} directs operations procedures and communicates with the other test personnel.
    \item The \primaryfull{} operates all manual valves for the fill system.
    \item The \secondaryfull{} is the backup for \primary{}, and communicates with OPS. If \primary{} becomes incapacitated, \secondary{} is responsible for removing them from danger.
    \item The \daqfull{} monitors and operates the test data acquisition system.
    \item the \controlfull{} operates the test control system, including actuation of remote valves and engine ignition.
    \item \perifull{}, \periifull{}, \periiifull{}, and perivfull{} ensure that no unauthorized personnel enter the testing area during test operations.
\end{checklist}
\setcounter{checklistnum}{0}
% end personnel required section -- }}}

% begin sign off -- {{{
\subsection{Sign-Off}
\textit{To be completed by all test personnel after reading and familiarization with procedures}
\begin{checklist}
    \item \opsfull{}      \tabto{25em}\rule{10em}{0.4pt}\hspace{5em}\rule{10em}{0.4pt}
    \item \primaryfull{}  \tabto{25em}\rule{10em}{0.4pt}\hspace{5em}\rule{10em}{0.4pt}
    \item \secondaryfull{}\tabto{25em}\rule{10em}{0.4pt}\hspace{5em}\rule{10em}{0.4pt}
    \item \daqfull{}      \tabto{25em}\rule{10em}{0.4pt}\hspace{5em}\rule{10em}{0.4pt}
    \item \controlfull{}  \tabto{25em}\rule{10em}{0.4pt}\hspace{5em}\rule{10em}{0.4pt}
    \item \perifull{}     \tabto{25em}\rule{10em}{0.4pt}\hspace{5em}\rule{10em}{0.4pt}
    \item \periifull{}    \tabto{25em}\rule{10em}{0.4pt}\hspace{5em}\rule{10em}{0.4pt}
    \item \periiifull{}   \tabto{25em}\rule{10em}{0.4pt}\hspace{5em}\rule{10em}{0.4pt}
    \item \perivfull{}    \tabto{25em}\rule{10em}{0.4pt}\hspace{5em}\rule{10em}{0.4pt}
\end{checklist}
\setcounter{checklistnum}{0}
% end sign off -- }}}

\newpage
% begin prior to start section -- {{{
\subsection{Prior to Start}
\begin{checklist}
    \item Ensure that the following procedures are complete:
    \begin{checklist}
        \item Combustion Chamber Assembly procedure
        \item Oxidizer Tank Assembly procedure
        \item Plumbing Setup procedure
        \item Oxidizer Tank Stand Setup procedure
        \item Tank Heating Setup procedure
        \item Test Stand Setup procedure
        \item Data Acquisition Setup procedure
        \item Test Control System Setup procedure
        \item Perimeter Checks procedure
    \end{checklist}
    \item Ensure that all personnel as defined above are available and have completed the sign-off.
    \item Ensure that the following personnel have walkie-talkies and communication is functional:
    \begin{checklist}
        \item \ops{}
        \item \secondary{}
        \item \daq{}
        \item \peri{}
        \item \perii{}
        \item \periii{}
        \item \periv{}
    \end{checklist}
    \item Ensure that all spectators and test personnel are wearing safety glasses and hearing protection.
    \item Ensure that \primary{} and \secondary{} are wearing face shields and have no exposed skin.
    \item Ensure that \primary{} is wearing thermal gloves.
    \item Ensure that \primary{} is in possession of the supply cylinder gasket. 
    \item Ensure that \secondary{} is in possession of a multimeter.
    \item Ensure that \ops{} is in possession of the system control key.
\end{checklist}
\setcounter{checklistnum}{0}
% end prior to start section -- }}}

% begin Fill System Check section -- {{{
\subsection{Fill System Check Procedure}
\begin{checklist}
    \item \secondary{}: Confirm that the ignition wires are not connected to the engine.
    \item \control{}: Actuate the Tank Heating Valve in order to test the tank heating system. 
    \item \daq{}: Confirm that the water temperature is increasing. 
    \item \control{}: Close the Tank Heating Valve. 
    \item \primary{}: Open the Tank Heating Drain Valve.
    \item \primary{}: Confirm that the following valves are initially closed:
    \begin{checklist}
    		\item Cylinder Valve (SC-1)
    		\item Remote Fill Valve (MV-1)
    		\item Parallel Fill Valve (BA-2)
    		\item Tank Vent Valve (MV-2)
    		\item Line Vent Valve (BA-3)
    		\item Injector Valve (IJ-1)
    	\end{checklist}
    	\item \primary{}: Confirm that the following valves are initially open:
    	\begin{checklist}
    		\item Series Fill Valve (BA-1)
    	\end{checklist}
    	\item \ops{}: Confirm that ops is in possession of the system control key.
    	\item \daq{}: Confirm that all pressure transducers are reading atmospheric. 
    	\item \daq{}: Confirm that all load cells are reading the determined zero point.
    	\item \peri{}, \perii{}, \periii{}, \periv{}: Close the perimeter and do not allow any further personnel to enter the testing area. 
    	\item \secondary{}: Confirm that no personnel are in the testing area other than \primary{} and \secondary{}.
    	\item \primary{}: Remove all covers from the plumbing:
    	\begin{checklist}
    		\item Tank Vent Valve
    		\item Pressure Relief Valve
    		\item Line Vent Valve
    	\end{checklist}
    	\item \primary{}: Remove the cap from the carbon dioxide supply cylinder. 
    	\item \primary{}: Connect the fill line to the supply cylinder with the gasket, hand tighten, and then tighten with a wrench. Do not force a connection. 
    	\item \primary{}: Slowly open the Cylinder Valve (SC-1) through $\frac{3}{4}$ of a turn. 
    	\begin{checklist}[label=$\bullet$]
    		\item If leaks are observed:
    		\begin{checklist}
    			\item \primary{}: Close the Cylinder Valve (SC-1). 
    			\item \primary{}: Slowly open the Line Vent Valve (BA-3). 
    			\item \primary{}: Slowly open the Parallel Fill Valve (BA-2).
    			\item \daq{}: Confirm that the Fill Pressure Transducer is reading atmospheric pressure. 
    			\item \ops{}: Abort test procedures and revisit the plumbing setup.
    		\end{checklist}
    	\end{checklist}
    	\item \primary{}: Communicate the supply cylinder pressure as visible on the Pressure Gauge. 
    	\item \daq{}: Communicate the supply cylinder pressure as read by the Fill Pressure Transducer. 
    	\item \daq{}: Confirm that the two measurements are in agreement. 
    	\item \ops{}: Give the system control key to \control{}.
    	\item \control{}: Engage the key switch and power on the control boxes. 
    	\item \control{}: Open the Tank Vent Valve (MV-2). 
    	\item \control{}: Open the Remote Fill Valve (MV-1). 
    	\begin{checklist}[label=$\bullet$]
    		\item If leaks are observed:
    		\begin{checklist}
    			\item \control{}: Close the Remote Fill Valve (MV-1). 
    			\item \primary{}: Close the Cylinder Valve (SC-1).
    			\item \primary{}: Slowly open the Line Vent Valve (BA-3). 
    			\item \primary{}: Slowly open the Parallel Fill Valve (BA-2).
    			\item \control{}: Open the Remote Fill Valve (MV-1). 
    			\item \daq{}: Confirm that the Fill Pressure Transducer is reading atmospheric pressure. 
    			\item \ops{}: Abort test procedures and revisit the plumbing setup. 
    		\end{checklist}
    		\item If the Remote Fill Valve fails to open:
    		\begin{checklist}
    			\item \primary{}: Close the Cylinder Valve (SC-1). 
    			\item \primary{}: Slowly open the Line Vent Valve (BA-3). 
    			\item \primary{}: Slowly open the Parallel Fill Valve (BA-2).
    			\item \daq{}: Confirm that the Fill Pressure Transducer is reading atmospheric pressure. 
    			\item \ops{}: Abort test procedures and revisit the plumbing setup.
    		\end{checklist}
    	\end{checklist}
    	\item \daq{}: Confirm that the oxidizer tank mass is increasing. 
    	\item \daq{}: Confirm that the oxidizer tank pressure is increasing. 
    	\item \control{}: Close the Remote Fill Valve (MV-1).
    	\item \primary{}: Open the Line Vent Valve (BA-3). 
    	\item \daq{}: Confirm that the Oxidizer Tank Pressure Transducer is reading atmospheric pressure. 
    	\item \primary{}: Close the Cylinder Valve (SC-1). 
    	\item \control{}: Open the Remote Fill Valve (MV-1). 
    	\item \daq{}: Confirm that the Fill Pressure Transducer is reading atmospheric pressure. 
    	\item \primary{}: Disconnect the fill line from the supply cylinder. 
    	\item \primary{}: Replace the cap on the carbon dioxide supply cylinder. 
    	\item \ops{}: Wait for at least 3 minutes before proceeding. 
    	\item \peri{}, \perii{}, \periii{}, \periv{}: Open the perimeter. 
    	\item \ops{}: Proceed with final setup for the Static Fire Test procedure. 
\end{checklist}
\setcounter{checklistnum}{0}
% end Fill System Check Section -- }}}
\newpage

%begin Prior to Static Fire Test Section -- {{{
\subsection{Prior to Static Fire Test}
\begin{checklist}
	\item Confirm that the nozzle is filled with water and not leaking.
	\item Confirm that there are no fire hazards within the testing area.
	\item Confirm that the cameras are set up at the correct locations. 
\end{checklist}
\setcounter{checklistnum}{0}
% end Prior to Static Fire Test Section -- }}}

%begin Static Fire Test Section -- {{{
\subsection{Static Fire Test - Remote Control Procedure}
\begin{checklist}
	\item \secondary{}: Confirm that the ignition wires are not connected to the engine. 
	\item \primary{}: Confirm that the following valves are initially closed:
	\begin{checklist}
		\item Cylinder Valve (SC-1)
		\item Remote Fill Valve (MV-1)
		\item Parallel Fill Valve (BA-2)
		\item Tank Vent Valve (MV-2)
		\item Line Vent Valve (BA-3)
		\item Injector Valve (IJ-1)
	\end{checklist}
	\item \primary{}: Confirm that the following valves are initially open:
	\begin{checklist}
		\item Series Fill Valve (BA-1)
	\end{checklist}
	\item \ops{}: Ensure that ops is in possession of the system control key. 
	\item \daq{}: Confirm that all pressure transducers are reading atmospheric pressure. 
	\item \daq{}: Confirm that all load cells are reading the determined zero point. 
	\item \daq{}: Confirm that all thermistors are reading ambient temperature. 
	\item \textbf{\textit{PAUSE POINT}}
	\item \peri{}, \perii{}, \periii{}, \periv{}: Close the perimeter and do not allow any further personnel to enter the testing area. 
	\item \secondary{}: Confirm that there are no personnel present in the testing area other than \primary{} and \secondary{}.
	\item \primary{}: Remove all covers from the plumbing:
	\begin{checklist}
		\item Tank Vent Valve
		\item Pressure Relief Valve
		\item Line Vent Valve
		\item Nozzle
	\end{checklist}
	\item \primary{}: Turn on the air compressor by adjusting the regulator to maximum. 
	\item \primary{}: Confirm that the pressure gauge on the air compressor is reading approximately 85 psi. 
	\item \primary{}: Pressurize the Injector Valve. 
	\item \daq{}: Confirm that the pressure switch for the Injector Valve is reading 0V.
	\item \secondary{}: Confirm that the resistance across the ignition coils is between 2.5 $\Omega$ and 3 $\Omega$:
	\begin{checklist}
		\item Primary ignition coil
		\item Secondary ignition coil
	\end{checklist}
	\item \secondary{}: Connect the ignition connectors to the RLCS ignition cable.
	\item \primary{}: Remove the cap from the nitrous oxide supply cylinder.
	\item \primary{}: Connect the fill line to the supply cylinder with the gasket, hand tighten, and then tighten with a wrench. Do not force the connection. 
	\item \primary{}: Slowly open the supply cylinder through $\frac{3}{4}$ of a turn. 
	\begin{checklist}[label=$\bullet$]
    		\item If leaks are observed:
    		\begin{checklist}
    			\item \primary{}: Close the Cylinder Valve (SC-1). 
    			\item \primary{}: Slowly open the Line Vent Valve (BA-3). 
    			\item \primary{}: Slowly open the Parallel Fill Valve (BA-2).
    			\item \daq{}: Confirm that the Fill Pressure Transducer is reading atmospheric pressure. 
    			\item \secondary{}: Disconnect the ignition connectors from the RLCS ignition cable. 
    			\item \primary{}: Turn off the air compressor and depressurize the Injector Valve. 
    			\item \ops{}: Abort test procedures and revisit the plumbing setup.
    		\end{checklist}
    	\end{checklist}
    	\item \primary{}: Communicate the supply cylinder pressure as visible on the Pressure Gauge. 
    	\item \daq{}: Communicate the supply cylinder pressure as read by the Fill Pressure Transducer. 
    	\item \daq{}: Confirm that the two pressure measurements are in agreement.
    	\item \primary{}: Turn on the camera.  
    	\item \primary{} and \secondary{}: Retreat to the mission control area. 
    	\item \control{}: Confirm that all actuator controls are in the "off" position:
    	\begin{checklist}
    		\item Remote Fill Valve (MV-1)
    		\item Tank Vent Valve (MV-2)
    		\item Injector Valve (IJ-1)
    		\item Primary Ignition
    		\item Secondary Ignition
    	\end{checklist}
    	\item \textbf{\textit{PAUSE POINT}}
    	\item \ops{}: Poll the following personnel for GO/NO GO status:
    	\begin{checklist}
    		\item \peri{}
    		\item \perii{}
    		\item \periii{}
    		\item \periv{}
    		\item \daq{}
    		\item \control{}
    		\item \primary{}
    		\item \secondary{}
    	\end{checklist}
    	\item \ops{}: Give the system control key to \control{}.
    	\item \control{}: Engage the key switch and power on the control boxes. 
    	\item \control{}: Open the Tank Vent Valve (MV-2).
    	\item \control{}: Open the Remote Fill Valve (MV-1).
    	\begin{checklist}[label=$\bullet$]
    		\item If leaks are observed:
    		\begin{checklist}
    			\item \control{}: Close the Remote Fill Valve (MV-1). 
    			\item \primary{}: Open the Line Vent Valve (BA-3) using the ropes. 
    			\item \ops{}: Proceed only when the oxidizer tank has fully vented. 
    			\item \primary{}: and \secondary{}: Approach the test plumbing.
    			\item \primary{}: Close the Cylinder Valve (SC-1). 
    			\item \control{}: Open the Remote Fill Valve (MV-1). 
    			\item \daq{}: Confirm that the Fill Pressure Transducer is reading atmospheric pressure. 
    			\item \secondary{}: Disconnect the ignition connectors from the RLCS ignition cable. 
    			\item \primary{}: Turn off the air compressor and depressurize the Injector Valve. 
    			\item \ops{}: Abort test procedures and revisit plumbing setup.
    		\end{checklist}
    		\item If the Remote Fill Valve fails to open:
    		\begin{checklist}
    			\item \primary{}: Close the Cylinder Valve (SC-1). 
    			\item \primary{}: Slowly open the Line Vent Valve (BA-3). 
    			\item \primary{}: Slowly open the Parallel Fill Valve (BA-2).
    			\item \daq{}: Confirm that the Fill Pressure Transducer is reading atmospheric pressure. 
    			\item \secondary{}: Disconnect the ignition connectors from the RLCS ignition cable. 
    			\item \primary{}: Turn off the air compressor and depressurize the Injector Valve. 
    			\item \ops{}: Abort test procedures and revisit the control system setup. 
    		\end{checklist}
    	\end{checklist}
    	\item \ops{}: Proceed only when a white plume is visible from the Tank Vent Valve (MV-2). 
    	\item \control{}: Close the Tank Vent Valve (MV-2).
    	\item \control{}: Close the Remote Fill Valve (MV-1). 
    	\begin{checklist}[label=$\bullet$]
    		\item If the Remote Fill Valve fails to close:
    		\begin{checklist}
    			\item \primary{} and \secondary{}: Approach the test plumbing. 
    			\item \primary{}: Close the Series Fill Valve (BA-1).
    			\item \primary{} and \secondary{}: Retreat to the mission control area. 
    		\end{checklist}
    	\end{checklist}
    	\item \control{}: Open the Tank Heating Valve. 
    	\item \daq{}: Proceed only when the oxidizer tank pressure is at least 750 psi. 
    	\begin{checklist}[label=$\bullet$]
    		\item If the oxidizer tank pressure does not reach 750 psi:
    		\begin{checklist}
    			\item \control{}: Close the Tank Heating Valve. 
    			\item \primary{}: Open the Line Vent Valve (BA-3) using the ropes.
    			\item \ops{}: Proceed only when the system has fully vented. 
    			\item \primary and \secondary{}: Approach the test plumbing. 
    			\item \primary{}: Close the Cylinder Valve (SC-1).
    			\item \control{}: Open the Tank Vent Valve (MV-2). 
    			\item \control{}: Open the Remote Fill Valve (MV-1). 
    			\item \daq{}: Confirm that the Oxidizer Tank Pressure Transducer is reading atmospheric pressure. 
    			\item \secondary{}: Disconnect the ignition connectors from the RLCS ignition cable. 
    			\item \primary{}: Turn off the air compressor and depressurize the Injector Valve. 
    			\item \ops{}: Abort test procedures and revisit the tank heating setup. 
    		\end{checklist}
    	\end{checklist}
    	\item \control{}: Close the Tank Heating Valve. 
    	\item \textbf{\textit{PAUSE POINT}}
    	\item \perii{}: Move to the viewing location. 
    	\item \control{}: Perform the engine startup procedure:
    	\begin{checklist}
    		\item Arm the Primary Ignition switch.
    		\item Hold down the Fire button until black smoke is observed. Continuously communicate the ignition current reading as displayed by the control box. 
    		\begin{checklist}[label=$\bullet$]
    			\item In the event of a failed ignition (smoke not observed within 1 minute):
    			\begin{checklist}
    				\item \control{}: Disarm the Primary Ignition Switch.
    				\item \control{}: Arm the Secondary Ignition Switch.
    				\item \ops{}: Revisit ignition setup.
    			\end{checklist}
    			\item In the event of a second failed ignition (smoke not observed within 1 minute):
    			\begin{checklist}
    				\item \control{}: Disarm the secondary ignition switch.
    				\item \primary{}: Open the Line Vent Valve (BA-3) using the ropes. 
    				\item \ops{}: Proceed only when the oxidizer tank is fully vented. 
    				\item \primary{} and \secondary{}: Approach the test plumbing. 
    				\item \primary{}: Close the Cylinder Valve (SC-1). 
    				\item \control{}: Open the Remote Fill Valve (MV-1). 
    				\item \control{}: Open the Tank Vent Valve (MV-2). 
    				\item \daq{}: Confirm that the Oxidizer Tank Pressure Transducer is reading atmospheric pressure. 
    				\item \secondary{}: Disconnect the ignition connectors from the RLCS ignition cable. 
    				\item \primary{}: Turn off the air compressor and depressurize the Injector Valve. 
    				\item \ops{}: Abort test procedures and proceed to teardown. 
    			\end{checklist}
    		\end{checklist}
    		\item \control{}: Start the engine by opening the Injector Valve. 
    	\end{checklist}
    	\item \textbf{ALL}: Observe the plume. 
    	\item \textbf{\textit{PAUSE POINT}}
    	\item \perii{}: Return to your assigned position. 
    	\item \ops{}: Wait for at least 3 minutes before proceeding. 
    	\item \daq{}: Confirm that the Oxidizer Tank Pressure Transducer is reading atmospheric pressure. 
    	\item \control{}: Open the Tank Vent Valve (MV-2).
    	\item \primary{} and \secondary{}: Approach the plumbing setup. 
    	\item \primary{}: Close the Cylinder Valve (SC-1). 
    	\item \control{}: Open the Remote Fill Valve (MV-1). 
    	\item \daq{}: Confirm that the Fill Pressure Transducer is reading atmospheric pressure. 
    	\item \primary{}: Disconnect the fill line from the supply cylinder. 
    	\item \primary{}: Replace the cap on the nitrous oxide cylinder. 
    	\item \primary{}: Turn off the air compressor and depressurize the Injector Valve. 
    	\item \ops{}: Wait at least 3 minutes before proceeding.
    	\item \daq{}: Confirm that the nozzle thermistors are reading below 100 $^\circ$C, unless suspected faulty. 
    	\item \peri{}, \perii{}, \periii{}, \periv{}: Open the perimeter. 
    	\item \daq{}: Continue to monitor the thermistor readings and inform \ops{} if the combustion chamber temperature exceeds 190 $^\circ$C.
    	\item \ops{}: Proceed with teardown and disassembly. 
\end{checklist}
\setcounter{checklistnum}{0}
% end Static Fire Test Procedure section -- }}}

\end{document}
	
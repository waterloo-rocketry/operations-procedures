\documentclass[letter]{article}

%% margins and fonts
\usepackage[margin=1in]{geometry}
\renewcommand{\familydefault}{\sfdefault}

% begin package imports -- {{{
\usepackage{amsmath}
\usepackage{graphicx}
\usepackage[dvipsnames]{xcolor}
\usepackage{tabto}
\usepackage{titling}
\usepackage[iso,german]{isodate}
% end package imports -- }}}

% begin set section title styles -- {{{
\usepackage{titlesec}
\titleformat{\section} {\setcounter{checklistnum}{0} \normalfont\Large\bfseries}{}{0em}{}[{\titlerule[0.8pt]}]
\titleformat{\subsection} {\setcounter{checklistnum}{0} \normalfont\large}{}{0em}{}[{\titlerule[0.6pt]}]
% end set section title styles -- }}}

% begin checklist symbols definition -- {{{
\usepackage{enumitem,amssymb}
\newcounter{checklistnum}
\setcounter{checklistnum}{0}
\DeclareRobustCommand{\checklistnumber}{\refstepcounter{checklistnum}\thechecklistnum}
\newlist{checklist}{itemize}{6}
\setlist[checklist,1]{
label={\color{gray}\checklistnumber}\hspace{2em}$\square$,
leftmargin=0em,
itemindent=2em
}
\setlist[checklist,2]{
label={\color{gray}\checklistnumber}\hspace{4em}$\square$,
leftmargin=0em,
itemindent=4em
}
\setlist[checklist,3]{
label={\color{gray}\checklistnumber}\hspace{6em}$\square$,
leftmargin=0em,
itemindent=6em
}
\setlist[checklist,4]{
label={\color{gray}\checklistnumber}\hspace{8em}$\square$,
leftmargin=0em,
itemindent=8em
}
\setlist[checklist,5]{
label={\color{gray}\checklistnumber}\hspace{10em}$\square$,
leftmargin=0em,
itemindent=10em
}
\setlist[checklist,6]{
label={\color{gray}\checklistnumber}\hspace{12em}$\square$,
leftmargin=0em,
itemindent=12em
}
% end checklist symbols definition -- }}}

% begin personnel macro -- {{{
\newcommand{\operator}[4]{%
  \expandafter\newcommand\csname #1\endcsname{{\color{#2}\textbf{#3}}\phantom{}}
  \expandafter\newcommand\csname #1full\endcsname{{\color{#2}\textbf{#4 [#3]}}\phantom{}}
}
% end personnel macro -- }}}

% begin flag macro -- {{{
\newcommand\flag[1]{
    \hfill \makebox(0,0){\hspace{1.7in}\textcolor{#1}{\rule{0.3in}{0.5in}}}
}
% end flag macro -- }}}

\pagenumbering{gobble}


\title{
\Huge Unnamed Liquid Rocket Engine\\
Static Fire 1\\
\vspace{1cm}
\Large Static Fire Test Operations Procedures}

%begin operators definitions -- {{{
\operator{ops}{Blue}{OPS}{Operations Director}
\operator{primary}{Fuchsia}{PRIMARY}{Primary Fill Operator}
\operator{secondary}{Emerald}{SECONDARY}{Secondary Fill Operator}
\operator{daq}{OliveGreen}{DAQ}{DAQ Technician}
\operator{control}{Maroon}{CONTROL}{Control System Operator}
%end operators definitions -- }}}}}

\begin{document}
%%%%%%%%%%%%%%%%%%%%%%%%%%%%%%%%%%%%%%%%%%%%%%%%%%%%%%%%%%%%%%%%%%%%%%%%%%%%%%%%%
% Waterloo Rocketry Standard template for operations procedures                 %
% Copyright claimed by Aaron Morrison (akmorris@uwaterloo.ca)                   %
% Any commercial entity wishing to use this template in any way must provide    %
% to Waterloo Rocketry, one two-four (24 pack) of a non-shitty                  %
% type of beer or an amount of Canadian dollars roughly equalling               %
% the current cost of such a purchase. If you do so, then use it                %
% however you want, and I claim in no way that it compiles properly,            %
% and I take no responsibilty for any loss you incurr from using                %
% this template, and make no promises to support you in any technical           %
% capacity for sustained use.                                                   %
%%%%%%%%%%%%%%%%%%%%%%%%%%%%%%%%%%%%%%%%%%%%%%%%%%%%%%%%%%%%%%%%%%%%%%%%%%%%%%%%%

% begin titlepage -- {{{
\begin{center}
\vspace*{7cm}
\hspace{7em}\includegraphics[width=30em]{common/mono_horizontal_standard}
\newline
\rule{50em}{2pt}

\vspace{1cm}
\thetitle

\vspace*{\fill}
Compiled on \today
\end{center}
\newpage
% end titlepage -- }}}


\section{Static Fire Test Operations Procedures}

% begin contents subsection -- {{{
\subsection{Contents}
This document contains one procedure:
\begin{itemize}
    \item The \textit{Static Fire Test} procedure comprises steps for operating the fill system and conducting a static fire of the engine.
\end{itemize}
% end contents subsection -- }}}

% begin personnel required section -- {{{
\subsection{Personnel Required}
The test operations team consists of nine personnel:
\begin{checklist}
    \item The \opsfull{} directs operations procedures and communicates with the other test personnel.
    \item The \primaryfull{} operates all manual valves for the fill system.
    \item The \secondaryfull{} is the backup for \primary{}, and communicates with OPS. If \primary{} becomes incapacitated, \secondary{} is responsible for removing them from danger.
    \item The \daqfull{} monitors and operates the test data acquisition system.
    \item The \controlfull{} operates the test control system, including actuation of remote valves and engine ignition.
\end{checklist}
\setcounter{checklistnum}{0}
% end personnel required section -- }}}

% begin sign off -- {{{
\subsection{Sign-Off}
\textit{To be completed by all test personnel after reading and familiarization with procedures}
\begin{checklist}
    \item \opsfull{}      \tabto{25em}\rule{10em}{0.4pt}\hspace{5em}\rule{10em}{0.4pt}
    \item \primaryfull{}  \tabto{25em}\rule{10em}{0.4pt}\hspace{5em}\rule{10em}{0.4pt}
    \item \secondaryfull{}\tabto{25em}\rule{10em}{0.4pt}\hspace{5em}\rule{10em}{0.4pt}
    \item \daqfull{}      \tabto{25em}\rule{10em}{0.4pt}\hspace{5em}\rule{10em}{0.4pt}
    \item \controlfull{}  \tabto{25em}\rule{10em}{0.4pt}\hspace{5em}\rule{10em}{0.4pt}
\end{checklist}
\setcounter{checklistnum}{0}
% end sign off -- }}}

\newpage
% begin prior to start section -- {{{
\subsection{Prior to Start}
\begin{checklist}
    \item Ensure that all personnel as defined above are available and have completed the sign-off.
    \item Ensure that the following personnel have walkie-talkies and communication is functional:
    \begin{checklist}
        \item \ops{}
        \item \secondary{}
        \item \daq{}
        \item \control{}
    \end{checklist}
    \item Ensure that all spectators and test personnel are wearing safety glasses and hearing protection.
    \item Ensure that \primary{} and \secondary{} are wearing face shields and have no exposed skin.
    \item Ensure that \primary{} is wearing thermal gloves.
    \item Ensure that \secondary{} is in possession of a multimeter.
    \item Ensure that \ops{} is in possession of the system control key.
\end{checklist}
\setcounter{checklistnum}{0}
% end prior to start section -- }}}

\newpage

% begin Remote Control Procedure section -- {{{
\subsection{Static Fire Test - Remote Control Procedure}
\begin{checklist}
    \item \secondary{}: Confirm that the ignition wires are not connected to the engine.
    \item \primary{}: Confirm that the following valves are initially closed:
    \begin{checklist}
        \item SC-1 (Oxidizer Supply Valve)
        \item TK-1 (Pressurant Supply Valve)
        \item BA-1 (Oxidizer Pressurant Shutoff Valve)
        \item BA-2 (Fuel Pressurant Shutoff Valve)
        \item BA-3 (Oxidizer Parallel Fill Valve)
        \item BA-5 (Oxidizer Fill Line Vent Valve)
        \item BA-8 (Oxidizer Tank Dump Valve)
        \item BA-9 (Fuel Tank Vent Valve)
        \item BA-10 (Pressurant Line Vent Valve)
        \item MV-1 (Oxidizer Motorized Fill Valve)
        \item MV-2 (Pressurant Remote Valve)
        \item MV-3 (Oxidizer Motorized Vent Valve)
        \item MV-4 (Fuel Injector Valve)
        \item IJ-1 (Oxidizer Injector Valve)
    \end{checklist}
    \item \primary{}: Confirm that the following valves are initially open:
    \begin{checklist}
        \item BA-4 (Oxidizer Series Fill Valve)
        \item BA-6 (Oxidizer Shutoff Valve)
        \item BA-7 (Fuel Shutoff Valve)
    \end{checklist}
    \item \primary{}: Confirm that CV-1 is adjusted to the lowest pressure setting.
    \item \daq{}: Confirm that all pressure transducers are reading atmospheric pressure.
    \item \daq{}: Confirm that all load cells are reading the determined zero point.
    \item \daq{}: Confirm that all thermistors are reading ambient temperature.
    \item \textbf{\textit{PAUSE POINT}}
    \item \secondary{}: Confirm that no personnel are present in the testing area other than \primary{} and \secondary{}.
    \item \primary{}: Remove all plastic plugs and covers from the plumbing:
    \begin {checklist}
        \item Oxidizer Tank Remote Vent Line
        \item Oxidizer Tank Dump Line
        \item Oxidizer Fill Vent Line
        \item Fuel Tank Vent Line
        \item Pressurant Vent Line
        \item Nozzle
    \end {checklist}
    \item \primary{}: Remove the cap from the pressurant supply cylinder.
	\item \primary{}: Connect the pressurant line to the cylinder, hand tighten, and tighten with a wrench.
	\item \primary{}: Slowly open TK-1, watching for leaks.
	\begin{checklist}[label=$\bullet$]
        \item If leaks are observed:
        \begin{checklist}
            \item \primary{}: Close TK-1.
            \item \primary{}: Adjust CV-1 until PI-1 shows at least 100 psi.
            \item \primary{}: Slowly open BA-10 to vent the pressurant lines.
            \item \primary{}: Close BA-10.
            \item \primary{}: Inspect the plumbing connections at the pressurant lines.
        \end{checklist}
    \end{checklist}
    \item \primary{}: Adjust CV-1 until PI-1 shows 600 psi.
	\begin{checklist}[label=$\bullet$]
        \item If leaks are observed:
        \begin{checklist}
            \item \primary{}: Close TK-1.
            \item \primary{}: Slowly open BA-10 to vent the pressurant lines.
            \item \primary{}: Close BA-10.
            \item \primary{}: Inspect the plumbing connections at the pressurant lines.
        \end{checklist}
    \end{checklist}
    \item \daq{}: Confirm that PT-5 reads 600 psi.
	\item \primary{}: Open BA-1.
	\item \primary{}: Open BA-2.
	\item \daq{}: Confirm that PT-1 and PT-3 read atmospheric pressure.
    \item \secondary{}: Confirm that the resistance across the ignition coils is between 2.5 $\Omega$ and 3 $\Omega$:
    \begin{checklist}
            \item Primary ignition coil
            \item Secondary ignition coil
    \end{checklist}
    \item \secondary{}: Connect the ignition connectors to the ignition box.
    \item \primary{}: Remove the cap from the nitrous oxide supply cylinder.
    \item \primary{}: Connect the fill line to the supply cylinder, hand tighten, and then tighten with a wrench. Do not force the connection.
    \item \primary{}: Slowly open SC-1.
    \begin{checklist}[label=$\bullet$]
        \item If leaks are observed:
        \begin{checklist}
            \item \primary{}: Close SC-1.
            \item \primary{}: Open BA-8.
            \item \primary{}: Slowly open BA-3.
            \item \daq{}: Confirm that PT-2 is reading atmospheric pressure.
            \item \primary{}: Inspect the plumbing connections at the oxidizer fill lines.
        \end{checklist}
    \end{checklist}
    \item \primary{}: Communicate the supply cylinder pressure as visible on the Pressure Gauge.
    \item \daq{}: Communicate the supply cylinder pressure as read by the Fill Pressure Transducer.
    \item \daq{}: Confirm that the two pressure measurements are in agreement.
    \item \primary{} and \secondary{}: Retreat to the test control area.
    \item \control{}: Confirm that all actuator controls are in the ``off'' position:
    \begin{checklist}
        \item Motorized Fill Valve
        \item Motorized Vent Valve
        \item Pressurant Remote Valve
        \item Injector Valve
        \item Primary Ignition
        \item Secondary Ignition
    \end{checklist}
    \item \textbf{\textit{PAUSE POINT}}
    \item \ops{}: Give the system control key to \control{}.
    \item \control{}: Engage the key switch and power on the control boxes.
    \item \control{}: Open the Motorized Vent Valve.
    \item \control{}: Open the Motorized Fill Valve.
    \begin{checklist}[label=$\bullet$]
        \item If leaks are observed:
        \begin{checklist}
            \item \control{}: Close the Motorized Fill Valve.
            \item \ops{}: Proceed only when the oxidizer tank has fully vented.
            \item \primary{} and \secondary{}: Approach the test plumbing.
            \item \primary{}: Close SC-1.
            \item \control{}: Open the Motorized Fill Valve.
            \item \daq{}: Confirm that PT-1 and PT-3 are reading atmospheric pressure.
            \item \ops{}: Abort test procedures and revisit plumbing setup.
        \end{checklist}
        \item If the Remote Fill Valve fails to open:
        \begin{checklist}
            \item \ops{}: Abort test procedures and revisit control system setup.
        \end{checklist}
    \end{checklist}
    \item \daq{}: Proceed only when the oxidizer tank mass reaches a steady state.
    \item \control{}: Close the Motorized Vent Valve.
    \item \control{}: Close the Motorized Fill Valve.
    \item \textbf{\textit{PAUSE POINT}}
    \item \control{}: Open the Pressurant Remote Valve.
    \item \daq{}: Proceed only when PT-1 and PT-3 read 600 psi.
    \item \control{}: Close the Pressurant Remote Valve.
    \item \textbf{\textit{PAUSE POINT}}
    \item \ops{}: Poll the following personnel for GO/NO GO status:
    \begin{checklist}
        \item \primary{}
        \item \secondary{}
        \item \daq{}
        \item \control{}
    \end{checklist}
    \item \control{}: Perform engine startup procedure:
    \begin{checklist}
        \item Arm the Primary Ignition switch.
        \item Hold down the Fire button until the Primary ignition current drops to 0 A.
        \begin{checklist}[label=$\bullet$]
            \item In the event of a failed ignition (current drop not observed within 1 minute):
            \begin{checklist}
                \item \control{}: Disarm the Primary Ignition switch.
                \item \control{}: Arm the Secondary Ignition switch.
                \item \ops{}: Revisit ignition procedure.
            \end{checklist}
            \item In the event of a second failed ignition (current drop not observed within 1 minute):
            \begin{checklist}
                \item \control{}: Disarm the Secondary Ignition switch.
                \item \control{}: Open the Motorized Vent Valve to vent the oxidizer tank.
                \item \ops{}: Proceed only when the oxidizer tank has fully vented.
				\item \daq{}: Confirm that PT-1 is reading atmospheric pressure.
                \item \primary{} and \secondary{}: Approach the test plumbing.
                \item \primary{}: Open BA-9 using the ropes to depressurize the fuel tank.
                \item \daq{}: Confirm that PT-3 is reading atmospheric pressure.
                \item \primary{}: Close SC-1.
                \item \control{}: Open the Motorized Fill Valve to vent the oxidizer supply lines.
                \item \daq{}: Confirm that PT-2 is reading atmospheric pressure.
				\item \primary{}: Close TK-1.
	            \item \primary{}: Slowly open BA-10 to vent the pressurant lines.
                \item \daq{}: Confirm that PT-5 is reading atmospheric pressure.
                \item \ops{}: Abort test procedures and proceed to teardown.
            \end{checklist}
        \end{checklist}
        \item \control{}: Start the engine by opening the Injector Valve.
    \end{checklist}
    \item \textbf{ALL}: Observe the plume.
    \item \textbf{\textit{PAUSE POINT}}
    \item \ops{}: Wait for at least 3 minutes before proceeding.
    \item \daq{}: Confirm that PT-1 and PT-3 are reading atmospheric pressure.
    \item \control{}: Open MV-3.
    \item \primary{} and \secondary{}: Approach the test plumbing.
    \item \primary{}: Close SC-1.
    \item \control{}: Open the Motorized Fill Valve to vent the oxidizer supply lines.
    \item \daq{}: Confirm that PT-2 is reading atmospheric pressure.
	\item \primary{}: Close TK-1.
	\item \primary{}: Slowly open BA-10 to vent the pressurant lines.
    \item \daq{}: Confirm that PT-5 is reading atmospheric pressure.
    \item \primary{}: Disconnect the fill line from the oxidizer supply cylinder.
    \item \primary{}: Replace the cap on the oxidizer supply cylinder.
    \item \primary{}: Disconnect the fill line from the pressurant supply cylinder.
    \item \primary{}: Replace the cap on the pressurant supply cylinder.
    \item \ops{}: Wait for at least 3 minutes before proceeding.
    \item \daq{}: Continue to monitor thermistor readings and inform \ops{} if the combustion chamber temperature exceeds 190 $^\circ$C.
    \item \ops{}: Proceed with teardown and disassembly.

\end{checklist}
\setcounter{checklistnum}{0}
% end Static Fire Test Procedure section -- }}}

\end{document}

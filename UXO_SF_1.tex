\documentclass[letter]{article}

%% margins and fonts
\usepackage[margin=1in]{geometry}
\renewcommand{\familydefault}{\sfdefault}

% begin package imports -- {{{
\usepackage{amsmath}
\usepackage{graphicx}
\usepackage[dvipsnames]{xcolor}
\usepackage{tabto}
\usepackage{titling}
\usepackage[iso,german]{isodate}
% end package imports -- }}}

% begin set section title styles -- {{{
\usepackage{titlesec}
\titleformat{\section} {\setcounter{checklistnum}{0} \normalfont\Large\bfseries}{}{0em}{}[{\titlerule[0.8pt]}]
\titleformat{\subsection} {\setcounter{checklistnum}{0} \normalfont\large}{}{0em}{}[{\titlerule[0.6pt]}]
% end set section title styles -- }}}

% begin checklist symbols definition -- {{{
\usepackage{enumitem,amssymb}
\newcounter{checklistnum}
\setcounter{checklistnum}{0}
\DeclareRobustCommand{\checklistnumber}{\refstepcounter{checklistnum}\thechecklistnum}
\newlist{checklist}{itemize}{6}
\setlist[checklist,1]{
label={\color{gray}\checklistnumber}\hspace{2em}$\square$,
leftmargin=0em,
itemindent=2em
}
\setlist[checklist,2]{
label={\color{gray}\checklistnumber}\hspace{4em}$\square$,
leftmargin=0em,
itemindent=4em
}
\setlist[checklist,3]{
label={\color{gray}\checklistnumber}\hspace{6em}$\square$,
leftmargin=0em,
itemindent=6em
}
\setlist[checklist,4]{
label={\color{gray}\checklistnumber}\hspace{8em}$\square$,
leftmargin=0em,
itemindent=8em
}
\setlist[checklist,5]{
label={\color{gray}\checklistnumber}\hspace{10em}$\square$,
leftmargin=0em,
itemindent=10em
}
\setlist[checklist,6]{
label={\color{gray}\checklistnumber}\hspace{12em}$\square$,
leftmargin=0em,
itemindent=12em
}
% end checklist symbols definition -- }}}

% begin personnel macro -- {{{
\newcommand{\operator}[4]{%
  \expandafter\newcommand\csname #1\endcsname{{\color{#2}\textbf{#3}}\phantom{}}
  \expandafter\newcommand\csname #1full\endcsname{{\color{#2}\textbf{#4 [#3]}}\phantom{}}
}
% end personnel macro -- }}}

% begin flag macro -- {{{
\newcommand\flag[1]{
    \hfill \makebox(0,0){\hspace{1.7in}\textcolor{#1}{\rule{0.3in}{0.5in}}}
}
% end flag macro -- }}}

\pagenumbering{gobble}


\title{
\Huge UXO Hybrid Rocket Engine\\
Static Fire 1\\
\vspace{1cm}
\Large Static Fire Test Operations Procedures}

%begin operators definitions -- {{{
\operator{ops}{Blue}{OPS}{Operations Director}
\operator{primary}{Fuchsia}{PRIMARY}{Primary Fill Operator}
\operator{secondary}{Emerald}{SECONDARY}{Secondary Fill Operator}
\operator{daq}{OliveGreen}{DAQ}{DAQ Technician}
\operator{heat}{orange}{HEAT}{Heating Technician}
\operator{peri}{red}{P1}{Perimeter Guard 1}
\operator{perii}{red}{P2}{Perimeter Guard 2}
%end operators definitions -- }}}}}

\begin{document}
\input{common/standard_titlepage}

\section{Static Fire Test Operations Procedures}

% begin contents subsection -- {{{
\subsection{Contents}
This document contains three procedures:
\begin{itemize}
    \item The \textit{Fill System Check} procedure comprises steps for validating the integrity of the system plumbing and correct operation of the test data acquisition system, using carbon dioxide.
    \item The \textit{Static Fire Test – Remote Control} procedure comprises steps for operating the fill system using the electrical control system and motorized ball valves.
    \item The \textit{Static Fire Test – Manual Control} procedure comprises steps for operating the fill system using manually operated ball valves.
\end{itemize}
% end contents subsection -- }}}

% begin personnel required section -- {{{
\subsection{Personnel Required}
The test operations team consists of seven personnel:
\begin{checklist}
    \item The \opsfull{} directs operations procedures and communicates with the other test personnel.
    \item The \primaryfull{} is the main system operator. \primary{} operates all manual valves as well as the test control system.
    \item The \secondaryfull{} is the backup for \primary{}, and communicates with OPS. If \primary{} becomes incapacitated, \secondary{} is responsible for removing them from danger.
    \item The \daqfull{} monitors and operates the test data acquisition system.
    \item The \heatfull{} operates the valves for the tank heating system.
    \item \perifull{} and \periifull{} ensure that no unauthorized personnel enter the testing area during test operations.
\end{checklist}
\setcounter{checklistnum}{0}
% end personnel required section -- }}}

% begin sign off -- {{{
\subsection{Sign-Off}
\textit{To be completed by all test personnel after reading and familiarization with procedures}
\begin{checklist}
    \item \opsfull      \tabto{25em}\rule{10em}{0.4pt}\hspace{5em}\rule{10em}{0.4pt}
    \item \primaryfull  \tabto{25em}\rule{10em}{0.4pt}\hspace{5em}\rule{10em}{0.4pt}
    \item \secondaryfull\tabto{25em}\rule{10em}{0.4pt}\hspace{5em}\rule{10em}{0.4pt}
    \item \daqfull      \tabto{25em}\rule{10em}{0.4pt}\hspace{5em}\rule{10em}{0.4pt}
    \item \heatfull     \tabto{25em}\rule{10em}{0.4pt}\hspace{5em}\rule{10em}{0.4pt}
    \item \perifull     \tabto{25em}\rule{10em}{0.4pt}\hspace{5em}\rule{10em}{0.4pt}
    \item \periifull    \tabto{25em}\rule{10em}{0.4pt}\hspace{5em}\rule{10em}{0.4pt}
\end{checklist}
\setcounter{checklistnum}{0}
% end sign off -- }}}

\newpage
% begin prior to start section -- {{{
\subsection{Prior to Start}
\begin{checklist}
    \item Ensure that the following procedures are complete:
    \begin{checklist}
        \item Combustion Chamber Assembly procedure
        \item Oxidizer Tank Assembly procedure
        \item Plumbing Setup procedure
        \item Oxidizer Tank Stand Setup procedure
        \item Tank Heating Setup procedure
        \item Test Stand Setup procedure
        \item Data Acquisition Setup procedure
        \item Test Control System Setup procedure
    \end{checklist}
    \item Ensure that all technicians as defined above are available and have completed the sign-off.
    \item Ensure that the following personnel have walkie-talkies and communication is functional:
    \begin{checklist}
        \item \ops{}
        \item \secondary
        \item \daq{}
        \item \heat
        \item \peri{}
        \item \perii{}
    \end{checklist}
    \item Ensure that all spectators and test personnel are wearing safety glasses and hearing protection.
    \item Ensure that \primary{} and \secondary{} are wearing face shields and have no exposed skin.
    \item Ensure that \primary{} is wearing thermal gloves.
    \item Ensure that \secondary{} is in possession of the system control key.
\end{checklist}
\setcounter{checklistnum}{0}
% end prior to start section -- }}}

\newpage

% begin Fill System Check section -- {{{
\subsection{Fill System Check Procedure}
\begin{checklist}
    \item \primary{}: Confirm that the following valves are initially closed:
    \begin{checklist}
        \item Cylinder Valve
        \item Remote Fill Valve
        \item Parallel Fill Valve
        \item Tank Vent Valve
        \item Pressure Relief Valve
        \item Line Vent Valve
        \item Injector Valve
    \end{checklist}
    \item \primary{}: Confirm that the following valves are initially open:
    \begin{checklist}
        \item Series Fill Valve
    \end{checklist}
    \item \daq: Confirm that all pressure transducers are reading atmospheric pressure.
    \item \daq: Confirm that all load cells are reading the determined zero point.
    \item \ops: Confirm that all personnel in the testing area are aware of the test.
    \item \peri{} and \perii: Close the perimeter and do not allow any further personnel to enter the testing area.
    \item \secondary: Confirm that no personnel are present in the testing area other than \primary{} and \secondary.
    \item \primary: Remove all plastic plugs and covers from the plumbing:
    \begin {checklist}
        \item Tank Vent Valve
        \item Pressure Relief Valve
        \item Line Vent Valve
        \item Nozzle
    \end {checklist}
    \item \primary: Remove the cap from the carbon dioxide supply cylinder.
    \item \primary: Connect the fill line to the supply cylinder, hand tighten, and then tighten with a wrench. Do not force the connection.
    \item \primary: Slowly open the Cylinder Valve through $\frac{3}{4}$ of a turn.
    \begin{checklist}[label=$\bullet$]
        \item If leaks are observed:
        \begin{checklist}
            \item \primary{}: Close the Cylinder Valve.
            \item \primary{}: Slowly open the Line Vent Valve.
            \item \primary{}: Slowly open the Parallel Fill Valve.
            \item \daq{}: Confirm that the Fill Pressure Transducer is reading atmospheric pressure.
            \item \ops{}: Abort test procedures and revisit plumbing setup.
        \end{checklist}
    \end{checklist}
    \item \primary: Communicate the supply cylinder pressure as visible on the Pressure Gauge.
    \item \daq: Communicate the supply cylinder pressure as read by the Fill Pressure Transducer.
    \item \daq: Confirm that the two pressure measurements are in agreement.
    \item \secondary: Give the system control key to \primary.
    \item \primary: Engage the key switch and power on the control boxes.
    \item \primary: Open the Tank Vent Valve.
    \item \primary: Open the Remote Fill Valve.
    \begin{checklist}[label=$\bullet$]
        \item If leaks are observed:
        \begin{checklist}
            \item \primary{}: Close the Remote Fill Valve.
            \item \primary{}: Close the Cylinder Valve.
            \item \primary{}: Slowly open the Line Vent Valve.
            \item \primary{}: Slowly open the Parallel Fill Valve.
            \item \primary{}: Open the Remote Fill Valve.
            \item \daq{}: Confirm that the Fill Pressure Transducer is reading atmospheric pressure.
            \item \ops{}: Abort test procedures and revisit plumbing setup.
        \end{checklist}
        \item If the Remove Fill Valve fails to open:
        \begin{checklist}
            \item \ops{}: Abort test procedures and revisit control system setup.
        \end{checklist}
    \end{checklist}
    \item \daq: Confirm that the oxidizer tank mass is increasing.
    \item \daq: Confirm that the oxidizer tank pressure is increasing.
    \item \primary: Close the Remote Fill Valve
    \item \primary{}: Open the Line Vent Valve.
    \item \daq{}: Confirm that the Oxidizer Tank Pressure Transducer is reading atmospheric pressure.
    \item \primary{} and \secondary: Approach the test plumbing.
    \item \primary{}: Close the Cylinder Valve.
    \item \primary{}: Open the Remote Fill Valve.
    \item \daq{}: Confirm that the Fill Pressure Transducer is reading atmospheric pressure.
    \item \primary{}: Disconnect the fill line from the supply cylinder.
    \item \primary{}: Replace the cap on the carbon dioxide supply cylinder.
    \item \ops{}: Wait for at least 3 minutes before proceeding.
    \item \peri{} and \perii{}: Open the perimeter.
    \item \ops{}: Proceed with teardown and disassembly. 
\end{checklist}
\setcounter{checklistnum}{0}
% end Fill System Check section -- }}}
\newpage

% begin Remote Control Procedure section -- {{{
\subsection{Static Fire Test - Remote Control Procedure}
\begin{checklist}
    \item \secondary: Confirm that the ignition wires are not connected to the engine.
    \item \primary{}: Confirm that the following valves are initially closed:
    \begin{checklist}
        \item Cylinder Valve
        \item Remote Fill Valve
        \item Parallel Fill Valve
        \item Pressure Relief Valve
        \item Tank Vent Valve
        \item Line Vent Valve
        \item Injector Valve
    \end{checklist}
    \item \primary{}: Confirm that the following valves are initially open:
    \begin{checklist}
        \item Series Fill Valve
    \end{checklist}
    \item \daq{}: Confirm that all pressure transducers are reading atmospheric pressure.
    \item \daq{}: Confirm that all load cells are reading the determined zero point.
    \item \textbf{\textit{PAUSE POINT}}
    \item \peri{} and \perii{}: Close the perimeter and do not allow any further personnel to enter the testing area.
    \item \secondary: Confirm that no personnel are present in the testing area other than \primary{} and \secondary.
    \item \primary: Remove all plastic plugs and covers from the plumbing:
    \begin {checklist}
        \item Tank Vent Valve
        \item Pressure Relief Valve
        \item Line Vent Valve
        \item Nozzle
    \end {checklist}
    \item \secondary: Confirm that the impedance across the ignition coils is between 2.5 $\Omega$ and 3 $\Omega$:
    \begin{checklist}
            \item Primary ignition coil
            \item Secondary ignition coil
    \end{checklist}
    \item \secondary: Connect the ignition connectors to the ignition box. 
    \item \primary{}: Remove the cap from the nitrous oxide supply cylinder.
    \item \primary{}: Connect the fill line to the supply cylinder, hand tighten, and then tighten with a wrench. Do not force the connection.
    \item \primary: Slowly open the Cylinder Valve through $\frac{3}{4}$ of a turn.
    \begin{checklist}[label=$\bullet$]
        \item If leaks are observed:
        \begin{checklist}
            \item \primary{}: Close the Cylinder Valve.
            \item \primary{}: Slowly open the Parallel Fill Valve.
            \item \primary: Open the Line Vent Valve using the ropes.
            \item \daq{}: Confirm that the Fill Pressure Transducer is reading atmospheric pressure.
            \item \ops{}: Abort test procedures and revisit plumbing setup.
        \end{checklist}
    \end{checklist}
    \item \primary{}: Communicate the supply cylinder pressure as visible on the Pressure Gauge.
    \item \daq{}: Communicate the supply cylinder pressure as read by the Fill Pressure Transducer.
    \item \daq{}: Confirm that the two pressure measurements are in agreement.
    \item \primary{} and \secondary: Retreat to the test control area, behind the blast shield.
    \item \primary{}: Confirm that all actuator controls are in the ``off'' position:
    \begin{checklist}
        \item Remote Fill Valve
        \item Tank Vent Valve
        \item Injector Valve
        \item Primary Ignition
        \item Secondary Ignition
    \end{checklist}
    \item \textbf{\textit{PAUSE POINT}}
    \item \ops{}: Poll the following personnel for GO/NO GO status:
    \begin{checklist}
        \item \peri{}
        \item \perii{}
        \item \heat
        \item \daq{}
        \item \primary{}
        \item \secondary
    \end{checklist}
    \item \secondary: Give the system control key to \primary{}.
    \item \primary{}: Engage the key switch and power on the control boxes.
    \item \primary{}: Open the Tank Vent Valve.
    \item \primary{}: Open the Remote Fill Valve.
    \begin{checklist}[label=$\bullet$]
        \item If leaks are observed:
        \begin{checklist}
            \item \primary{}: Close the Remote Fill Valve.
            \item \primary{}: Open the Line Vent Valve using the ropes.
            \item \secondary: Proceed only when the oxidizer tank has fully vented.
            \item \primary{} and \secondary: Approach the test plumbing.
            \item \primary{}: Close the Cylinder Valve.
            \item \primary{}: Open the Remote Fill Valve.
            \item \daq{}: Confirm that the Fill Pressure Transducer is reading atmospheric pressure.
            \item \ops{}: Abort test procedures and revisit plumbing setup.
        \end{checklist}
        \item If the Remote Fill Valve fails to open:
        \begin{checklist}
            \item \ops{}: Proceed to the Manual Control procedure.
        \end{checklist}
    \end{checklist}
    \item \secondary{}: Proceed only when a white plume is visible from the Tank Vent Valve.
    \item \primary{}: Close the Tank Vent Valve.
    \item \primary{}: Close the Remote Fill Valve.
    \begin{checklist}[label=$\bullet$]
        \item If the Remote Fill Valve fails to close:
        \begin{checklist}
            \item \primary{} and \secondary: Approach the test plumbing.
            \item \primary{}: Close the Series Fill Valve.
            \item \primary{} and \secondary: Retreat to the test control area, behind the blast shield.
        \end{checklist}
    \end{checklist}
    \item \heat: Open the Tank Heating Valve.
    \item \daq{}: Proceed only when the oxidizer tank pressure is at least 750 psi.
    \begin{checklist}[label=$\bullet$]
        \item If the oxidizer tank pressure does not reach 750 psi:
        \begin{checklist}
            \item \heat: Close the Tank Heating Valve.
            \item \primary{}: Open the Line Vent Valve using the ropes.
            \item \secondary: Proceed only when the oxidizer tank has fully vented.
            \item \primary{} and \secondary: Approach the test plumbing.
            \item \primary{}: Close the Cylinder Valve.
            \item \primary{}: Open the Tank Vent Valve.
            \item \primary{}: Open the Remote Fill Valve.
            \item \daq{}: Confirm that the Oxidizer Tank Pressure Transducer is reading atmospheric pressure.
            \item \ops{}: Abort test procedures and revisit water jacket setup.
        \end{checklist}
    \end{checklist}
    \item \heat: Close the Tank Heating Valve.
    \item \textbf{\textit{PAUSE POINT}}
    \item \primary{}: Perform ignition procedure:
    \begin{checklist}
        \item Arm the Primary Ignition switch.
        \item Hold down the Fire button until black smoke is observed.
        \begin{checklist}[label=$\bullet$]
            \item In the event of a failed ignition (smoke not observed within 1 minute):
            \begin{checklist}
                \item \primary: Disarm the Primary Ignition switch.
                \item \primary: Arm the Secondary Ignition switch.
                \item \ops: Revisit ignition procedure.
            \end{checklist}
            \item In the event of a second failed ignition (smoke not observed within 1 minute):
            \begin{checklist}
                \item \primary: Disarm the Secondary Ignition switch.
                \item \primary: Open the Line Vent Valve using the ropes.
                \item \ops: Proceed only when the oxidizer tank has fully vented.
                \item \primary{} and \secondary: Approach the test plumbing.
                \item \primary{}: Close the Cylinder Valve.
                \item \primary{}: Open the Remote Fill Valve.
                \item \primary{}: Open the Tank Vent Valve.
                \item \daq{}: Confirm that the Oxidizer Tank Pressure Transducer is reading atmospheric pressure.
                \item \ops: Abort test procedures and proceed to teardown.
            \end{checklist}
        \end{checklist}
    \end{checklist}
    \item \primary: Start the engine by opening the Injector Valve.
    \item \primary: Observe the plume:
    \begin{checklist}[label=$\bullet$]
        \item If any unexpected events occur during the engine firing:
        \begin{checklist}
            \item \primary{}: Open the Line Vent Valve using the ropes.
            \item \primary{}: Wait for 3 seconds.
            \item \primary{}: Close the Injector Valve.
        \end{checklist}
    \end{checklist}
    \item \textbf{\textit{PAUSE POINT}}
    \item \ops{}: Wait for at least 3 minutes before proceeding.
    \item \daq{}: Confirm that the Oxidizer Tank Pressure Transducer is reading atmospheric pressure.
    \item \primary{}: Open the Tank Vent Valve.
    \item \primary{} and \secondary: Approach the test plumbing.
    \item \primary{}: Close the Cylinder Valve.
    \item \primary{}: Open the Remote Fill Valve.
    \item \daq{}: Confirm that the Fill Pressure Transducer is reading atmospheric pressure.
    \item \primary{}: Disconnect the fill line from the supply cylinder.
    \item \primary{}: Replace the cap on the nitrous oxide supply cylinder.
    \item \ops{}: Wait for at least 3 minutes before proceeding.
    \item \peri{} and \perii{}: Open the perimeter.
    \item \ops{}: Proceed with teardown and disassembly.

\end{checklist}
\setcounter{checklistnum}{0}
% end Remote Control Procedure section -- }}}

\newpage
    
% begin Manual Control Procedure section -- {{{
\subsection{Static Fire Test - Manual Control Procedure}
\begin{checklist}
    \item \secondary: Confirm that the ignition wires are not connected to the engine.
    \item \primary{}: Confirm that the following valves are initially closed:
    \begin{checklist}
        \item Cylinder Valve
        \item Remote Fill Valve
        \item Parallel Fill Valve
        \item Pressure Relief Valve
        \item Line Vent Valve
        \item Injector Valve
        \item Series Fill Valve
    \end{checklist}
    \item \primary{}: Confirm that the following valves are initially open:
    \begin{checklist}
        \item Tank Vent Valve
    \end{checklist}
    \item \daq{}: Confirm that all pressure transducers are reading atmospheric pressure.
    \item \daq{}: Confirm that all load cells are reading the determined zero point.
    \item \textbf{\textit{PAUSE POINT}}
    \item \peri{} and \perii{}: Close the perimeter and do not allow any further personnel to enter the testing area.
    \item \secondary: Confirm that no personnel are present in the testing area other than \primary{} and \secondary.
    \item \primary: Remove all plastic plugs and covers from the plumbing:
    \begin {checklist}
        \item Tank Vent Valve
        \item Pressure Relief Valve
        \item Line Vent Valve
        \item Nozzle
    \end {checklist}
    \item \secondary: Confirm that the impedance across the ignition coils is between 2.5 $\Omega$ and 3 $\Omega$:
    \begin{checklist}
            \item Primary ignition coil
            \item Secondary ignition coil
    \end{checklist}
    \item \secondary: Connect the ignition connectors to the ignition box.
    \item \primary{}: Remove the cap from the nitrous oxide supply cylinder.
    \item \primary{}: Connect the fill line to the supply cylinder, hand tighten, and then tighten with a wrench. Do not force the connection.
    \item \primary: Slowly open the Cylinder Valve through $\frac{3}{4}$ of a turn.
    \begin{checklist}[label=$\bullet$]
        \item If leaks are observed:
        \begin{checklist}
            \item \primary{}: Close the Cylinder Valve.
            \item \primary{}: Slowly open the Parallel Fill Valve.
            \item \primary: Open the Line Vent Valve using the ropes.
            \item \daq{}: Confirm that the Fill Pressure Transducer is reading atmospheric pressure.
            \item \ops{}: Abort test procedures and revisit plumbing setup.
        \end{checklist}
    \end{checklist}
    \item \primary{}: Communicate the supply cylinder pressure as visible on the Pressure Gauge.
    \item \daq{}: Communicate the supply cylinder pressure as read by the Fill Pressure Transducer.
    \item \daq{}: Confirm that the two pressure measurements are in agreement.
    \item \secondary: Confirm that the following actuator controls are in the ``off'' position:
    \begin{checklist}
        \item Primary Ignition
        \item Secondary Ignition
    \end{checklist}
    \item \textbf{\textit{PAUSE POINT}}
    \item \ops{}: Poll the following personnel for GO/NO GO status:
    \begin{checklist}
        \item \peri{}
        \item \perii{}
        \item \heat
        \item \daq{}
        \item \primary{}
        \item \secondary
    \end{checklist}
    \item \primary{}: Open the Parallel Fill Valve.
    \begin{checklist}[label=$\bullet$]
        \item If leaks are observed:
        \begin{checklist}
            \item \primary{}: Close the Parallel Fill Valve.
            \item \primary{}: Open the Line Vent Valve using the ropes.
            \item \primary{}: Close the Cylinder Valve.
            \item \primary{}: Open the Parallel Fill Valve.
            \item \daq{}: Confirm that the Fill Pressure Transducer is reading atmospheric pressure.
            \item \ops{}: Abort test procedures and revisit plumbing setup.
        \end{checklist}
    \end{checklist}
    \item \secondary{}: Proceed only when a white plume is visible from the Tank Vent Valve.
    \item \primary{}: Close the Parallel Fill Valve.
    \item \primary{} and \secondary: Retreat to the test control area, behind the blast shield.
    \item \heat: Open the Tank Heating Valve.
    \item \daq{}: Proceed only when the oxidizer tank pressure is at least 750 psi.
    \begin{checklist}[label=$\bullet$]
        \item If the oxidizer tank pressure does not reach 750 psi:
        \begin{checklist}
            \item \heat: Close the Tank Heating Valve.
            \item \primary{}: Open the Line Vent Valve using the ropes.
            \item \secondary: Proceed only when the oxidizer tank has fully vented.
            \item \primary{} and \secondary: Approach the test plumbing.
            \item \primary{}: Close the Cylinder Valve.
            \item \primary{}: Slowly open the Parallel Fill Valve.
            \item \daq{}: Confirm that the Oxidizer Tank Pressure Transducer is reading atmospheric pressure.
            \item \ops{}: Abort test procedures and revisit water jacket setup.
        \end{checklist}
    \end{checklist}
    \item \heat: Close the Tank Heating Valve.
    \item \textbf{\textit{PAUSE POINT}}
    \item \primary{}: Perform ignition procedure:
    \begin{checklist}
        \item Arm the Primary Ignition switch.
        \item Hold down the Fire button until black smoke is observed.
        \begin{checklist}[label=$\bullet$]
            \item In the event of a failed ignition (smoke not observed within 1 minute):
            \begin{checklist}
                \item \primary: Disarm the Primary Ignition switch.
                \item \primary: Arm the Secondary Ignition switch.
                \item \ops: Revisit ignition procedure.
            \end{checklist}
            \item In the event of a second failed ignition (smoke not observed within 1 minute):
            \begin{checklist}
                \item \primary: Disarm the Secondary Ignition switch.
                \item \primary: Open the Line Vent Valve using the ropes.
                \item \ops: Proceed only when the oxidizer tank has fully vented.
                \item \primary{} and \secondary: Approach the test plumbing.
                \item \primary{}: Close the Cylinder Valve.
                \item \primary{}: Open the Remote Fill Valve.
                \item \primary{}: Open the Tank Vent Valve.
                \item \daq{}: Confirm that the Oxidizer Tank Pressure Transducer is reading atmospheric pressure.
                \item \ops: Abort test procedures and proceed to teardown.
            \end{checklist}
        \end{checklist}
    \end{checklist}
    \item \primary: Start the engine by opening the Injector Valve with the ropes.
    \item \primary: Observe the plume:
    \begin{checklist}[label=$\bullet$]
        \item If any unexpected events occur during the engine firing:
        \begin{checklist}
            \item \primary{}: Open the Line Vent Valve using the ropes.
        \end{checklist}
    \end{checklist}
    \item \textbf{\textit{PAUSE POINT}}
    \item \ops{}: Wait for at least 3 minutes before proceeding.
    \item \daq{}: Confirm that the Oxidizer Tank Pressure Transducer is reading atmospheric pressure.
    \item \primary{} and \secondary: Approach the test plumbing.
    \item \primary{}: Close the Cylinder Valve.
    \item \primary{}: Open the Parallel Fill Valve.
    \item \daq{}: Confirm that the Fill Pressure Transducer is reading atmospheric pressure.
    \item \primary{}: Disconnect the fill line from the supply cylinder.
    \item \primary{}: Replace the cap on the nitrous oxide supply cylinder.
    \item \ops{}: Wait for at least 3 minutes before proceeding.
    \item \peri{} and \perii{}: Open the perimeter.
    \item \ops{}: Proceed with teardown and disassembly.


\end{checklist}
% end Manual Control Procedure section -- }}}

\end{document}
